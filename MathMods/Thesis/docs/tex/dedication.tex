% !TeX root = thesis.tex
% !TeX spellcheck = en_GB
% !TeX encoding = UTF-8


I would like to dedicate this thesis to the beauty of Mathematics. I cannot express this feeling any more elegantly than Ito, who in a speech given marking his Kyoto prize in 1998, gives a wonderful description of mathematical beauty.

\begin{displayquote}
	In precisely built mathematical structures, mathematicians find the same sort of beauty others find in enchanting pieces of music, or in magnificent architecture. There is, however, one great difference between the beauty of mathematical structures and that of great art. Music by Mozart, for instance, impresses greatly even those who do not know musical theory; the cathedral in Cologne overwhelms spectators even if they know nothing about Christianity. The beauty in mathematical structures, however, cannot be appreciated without understanding of a group of numerical formulae that express laws of logic. Only mathematicians can read “musical scores” containing many numerical formulae, and play that “music” in their hearts. Accordingly, I once believed that without numerical formulae, I could never communicate the sweet melody played in my heart. Stochastic differential equations, called “to Formula,” are currently in wide use for describing phenomena of random fluctuations over time. When I first set forth stochastic differential equations, however, my paper did not attract attention. It was over ten years after my paper that other mathematicians began reading my “musical scores” and playing my “music” with their “instruments.” By developing my “original musical scores” into more elaborate “music,” these researchers have contributed greatly to developing “Ito Formula”.
\end{displayquote}

I would also like to thank Prof Antonelli for giving me a glimpse of this beauty.


%%% Local Variables:
%%% mode: latex
%%% TeX-master: t
%%% End:
