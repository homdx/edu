% !TeX root = ../thesis.tex
% !TeX spellcheck = en_GB
% !TeX encoding = UTF-8


\section{Introduction}
\label{sec:sp-asian-intro}
As we have seen in the earlier chapters, European and American options may be priced using the CRR and BS models. Even though the BS model has a very high degree of computability, it does not allow us to find closed-form pricing formulae for many path-dependent options. The way out is by using numerical methods. A class of numerical methods use tree or lattice methods in the CRR model. One simple idea is to apply an explicit pricing scheme using CRR, which tends to BS as the number of time steps increases to infinity. But the exponential number of paths ($2^n$ to be exact, where $n$ is the number of time steps) make the method very slow and memory intensive, making it impractical in terms of computability. A logical step would be to modify the basic CRR model to allow for approximations. In this direction, Gaudenzi et al\cite{Gaudenzi2010} introduced a new method called the `singular points method' for pricing certain path-dependent options in an efficient manner. The chapter is a study on this method.

We will mainly focus on Asian options, in which the price is expressed as a function of some form of averaging on the underlying's price. Popular Asian options use the arithmetic or geometric means as the average. Again, Asian options may be exercised only at maturity (European) or at any time till the maturity (American). They may give the owner of the option the right to either sell (put) or buy (call). Theoretically, we may study either a call or a put, because the framework for the other one may be derived in the exact same way.

TODO: Push this defn to an earlier chapter.
\begin{dfn}[Path-dependent option]
	A path-dependent option is an option for which the value of the option is dependent not only on the final value of the underlying, but also on the path taken to reach that value.
\end{dfn}



\section{Existing methods}
\label{sec:existing-methods}

Before we go into the details of the singular points method, we shall look into the pre-existing methods, and discuss their advantages and disadvantages briefly.

Ref: Forsyth et al (2002)

\begin{itemize}
\item American Asian options with arithmetic mean
	\begin{itemize}
	\item Tree based
		\begin{itemize}
		\item CRR binomial method
		\item Hull and White (1993)
		\item Barraquand and Pudet (1996)
		\item Chalasani et al (1999a, b)
		\end{itemize}
	\item PDE based
		\begin{itemize}
		\item Vecer (2001)
		\item D’Halluin et al (2005)
		\end{itemize}
	\end{itemize}
\item American lookback options
	\begin{itemize}
	\item Hull and White (1993)
	\item Barraquand and Pudet (1996)
	\item Babbs (2000), using a `change of numeraire' approach, which cannot be applied to the fixed-strike case
	\end{itemize}
\end{itemize}

TODO: Discuss the advantages and disadvantages of each method.

A number of these algorithms has been implemented in Premia 13. Premia is a software designed for option pricing, hedging and financial model calibration. It has been developed by the `MathFi' team in INRIA. It is provided with it's C/C++ source code and an extensive scientific documentation. More information about Premia can be found at \url{https://www.rocq.inria.fr/mathfi/Premia/}.



\section{The Singular Points method}
\label{sec:sp-asian-method}

The price of an Asian option at each instance is a continuous function of the underlying's average. Since the number of paths to a node in a binomial tree is finite, we have that at each node of the underlying's binomial tree, the option price may be represented as a piecewise-linear, continuous, convex function of the average. We shall develop the theoretical idea in this section. In the subsequent section, we shall see that the nature of the function allows us to make approximations with \emph{a priori} error bounds.


\begin{dfn}[Singular points and singular values] \label{def:sp-asian-sp}
	Let $ P = (P_i)_{i \in [n]} = ( (x_i, y_i) )_{i \in [n]} $, $ n \in \mathbb{N} $ be a sequence of points such that
	\begin{subequations} \label{eq:sp-asian-conditions}
		\begin{align}
			a =& x_0 < x_1 < \dots < x_{n-1} < x_n = b \\
			\label{eq:sp-asian-condition-slope}
			m_{i+1} :=& \frac{y_{i+1} - y_{i}}{x_{i+1} - x_{i}} \le \frac{y_{i+2} - y_{i+1}}{x_{i+2} - x_{i+1}} = m_{i+2} \qquad \forall i \in \{ 1, \dots, n-1 \}
		\end{align}
	\end{subequations}
	
	Let $ f:[a,b] \to [0, \infty) $ be the function obtained by linear interpolation of the points in $P$. From the definition of $f$ and	 \ref{eq:sp-asian-condition-slope}, the function is continuous, piecewise-linear and convex.
	
	Then, the elements of $P$ are called \emph{singular points of $f$} and the abscissae $ \{ x_i \}_{i \in [n]} $ are called \emph{singular values of $f$}.
\end{dfn}


\begin{rem} \label{rem:sp-asian-characterisation}
	We note that the singular points characterise such a function completely. This can be seen from the following representation of the function.
	\begin{equation}
		\label{eq:sp-asian-function-repr}
		f(x) = y_0 + \sum_{i=1}^n [ m_i ( \min \{x_{i}, x \} - \min \{ x_{i-1}, x \} ) ]
	\end{equation}
	Where $ m_{i+1} = \frac{y_{i+1} - y_{i}}{x_{i+1} - x_{i}} $ represents the slope of the function between $ (x_{i}, y_{i}) $ and $ (x_{i+1}, y_{i}) $.
\end{rem}

\begin{rem}
	From the conditions \ref{eq:sp-asian-conditions}, we get
	\begin{equation*}
		y_0 < y_1 < \dots < y_{n-1} < y_n
	\end{equation*}
	So it is equivalent to sort points using either abscissae or ordinates.
\end{rem}



\subsection{Upper estimates}
\label{subsec:sp-asian-upper-estimates}

The following lemmas shall provide us with the necessary framework for upper and lower estimates for approximations on the functions generated by singular points.

\begin{lmm}[Upper estimate]
	\label{lmm:sp-asian-upper-estimate}
	Let $ f:[a,b] \to [0, \infty) $ be a continuous, piecewise-linear, convex function characterised by the singular points $ P = ( (x_i, y_i) )_{i \in [n]} $. Then, if a point $ (x_j, y_j), j \in \{ 1, \dots, n-1\} $ is removed from the sequence, the function $ f_u: [a,b] \to [0, \infty) $ obtained by the new sequence $ (P_i)_{i \in [n] \setminus \{ j \}} $ is also continuous, piecewise-linear and convex, and
	\begin{equation}
		f_u(x) \ge f(x) \qquad \forall x \in [a,b]
	\end{equation}
\end{lmm}

\begin{proof}
	By construction, $ \forall x \notin ( x_{j-1} , x_{j+1} ), \; f_u(x) = f(x) $.
	
	Again, by construction, $ \forall x \in ( x_{j-1} , x_{j+1} ), f_u(x) = (1-t) f(x_{j-1}) + t f(x_{j+1}) $, where $ t = \frac{ x - x_{j-1} }{ x_{j+1} - x_{j-1} } $.
	
	Now, we have:
	\begin{alignat*}{9}
		          && x_{j-1}  & <  \qquad x          && <  x_{j+1} \\
		\implies  &&       0  & <  \quad x - x_{j-1} && <  x_{j+1} - x_{j-1} \\
		\implies  &&       0  & <  \frac{ x - x_{j-1} }{ x_{j+1} - x_{j-1} } && <  1 \\
		\implies  &&       0  & <  \qquad t          && <  1
	\end{alignat*}
	
	$f$ is convex $\implies \forall t \in (0,1), \; f( (1-t) x_{j-1} + t x_{j+1} ) < (1-t) f(x_{j-1}) + t f(x_{j+1}) $.
	
	Thus, $ f_u(x) \ge f(x) \; \forall x \in [a,b]$.
\end{proof}

\begin{figure}
	\centering
	
	\definecolor{ffxfqq}{rgb}{1.,0.4980392156862745,0.}
	\definecolor{ffqqqq}{rgb}{1.,0.,0.}
	\definecolor{cqcqcq}{rgb}{0.7529411764705882,0.7529411764705882,0.7529411764705882}
	\definecolor{yqyqyq}{rgb}{0.5019607843137255,0.5019607843137255,0.5019607843137255}
	\definecolor{zzttqq}{rgb}{0.6,0.2,0.}
	\definecolor{eqeqeq}{rgb}{0.8784313725490196,0.8784313725490196,0.8784313725490196}
	
	\begin{tikzpicture}[line cap=round,line join=round,>=triangle 45,x=0.8cm,y=0.8cm]
		\draw [color=eqeqeq,dotted, xstep=1.6cm,ystep=1.6cm] (0.,0.) grid (17.,13.);
		\draw[->,color=black] (0.,0.) -- (17.,0.);
		\foreach \x in {,2.,4.,6.,8.,10.,12.,14.,16.}
		\draw[shift={(\x,0)},color=black] (0pt,2pt) -- (0pt,-2pt);
		\draw[color=black] (16.66408269434807,0.08316246397131088) node [anchor=south west] { x};
		\draw[->,color=black] (0.,0.) -- (0.,13.);
		\foreach \y in {,2.,4.,6.,8.,10.,12.}
		\draw[shift={(0,\y)},color=black] (2pt,0pt) -- (-2pt,0pt);
		\draw[color=black] (0.10395306283371611,12.541470198511039) node [anchor=west] { y};
		\clip(0.,0.) rectangle (17.,13.);
		\draw [line width=1.2pt,color=yqyqyq] (8.,5.)-- (12.,8.);
		\draw [line width=1.2pt,color=yqyqyq] (16.,12.)-- (12.,8.);
		\draw [line width=1.2pt,color=yqyqyq] (8.,5.)-- (4.,3.);
		\draw [line width=1.2pt,color=yqyqyq] (4.,3.)-- (1.,2.);
		\draw [line width=0.4pt,color=cqcqcq] (12.,8.)-- (12.,0.);
		\draw [line width=0.4pt,color=cqcqcq] (8.,5.)-- (8.,0.);
		\draw [line width=0.4pt,color=cqcqcq] (4.,3.)-- (4.,0.);
		\draw [line width=0.4pt,color=cqcqcq] (1.,2.)-- (1.,0.);
		\draw [dash pattern=on 2pt off 2pt,color=ffqqqq] (4.,3.)-- (12.,8.);
		\draw [color=ffxfqq] (8.,5.)-- (8.,5.5);
		\draw [line width=0.4pt,color=cqcqcq] (16.,12.)-- (16.,0.);
		\begin{scriptsize}
			\draw [fill=zzttqq] (16.,12.) circle (1.5pt);
			\draw[color=zzttqq] (16.227479830446466,11.294033238941376) node {$P_5$};
			\draw [fill=zzttqq] (12.,8.) circle (1.5pt);
			\draw[color=zzttqq] (12.19410099249828,7.239863120339971) node {$P_4$};
			\draw [fill=zzttqq] (8.,5.) circle (1.5pt);
			\draw[color=zzttqq] (8.223093992250325,4.474711193293883) node {$P_3$};
			\draw [fill=zzttqq] (4.,3.) circle (1.5pt);
			\draw[color=zzttqq] (4.2105057668688834,2.4788120579824215) node {$P_2$};
			\draw [fill=zzttqq] (1.,2.) circle (1.5pt);
			\draw[color=zzttqq] (1.21665755725786,1.647187418269313) node {$P_1$};
			\draw [fill=zzttqq] (16.,0.) circle (1.5pt);
			\draw[color=zzttqq] (16.18589860531298,0.5452847706494437) node {$x_5$};
			\draw [fill=zzttqq] (12.,0.) circle (1.5pt);
			\draw[color=zzttqq] (12.152519767364792,0.524494154656616) node {$x_4$};
			\draw [fill=zzttqq] (8.,0.) circle (1.5pt);
			\draw[color=zzttqq] (8.160722154550095,0.5037035386637883) node {$x_3$};
			\draw [fill=zzttqq] (4.,0.) circle (1.5pt);
			\draw[color=zzttqq] (4.2105057668688834,0.5037035386637883) node {$x_2$};
			\draw [fill=zzttqq] (1.,0.) circle (1.5pt);
			\draw[color=zzttqq] (1.1958669446911165,0.5660753866422714) node {$x_1$};
			\draw [fill=ffqqqq] (8.,5.5) circle (1.5pt);
			\draw[color=ffqqqq] (7.99439725401615,6.075588624741618) node {$Q_3$};
		\end{scriptsize}
	\end{tikzpicture}
	
	\caption{Upper estimate: Illustration of Lemma \ref{lmm:sp-asian-upper-estimate} with $ j = 3 $}
	\label{fig:upper-estimate}
\end{figure}



\subsection{Lower estimates}
\label{subsec:sp-asian-lower-estimates}

\begin{lmm}[Lower estimate]
	\label{lmm:sp-asian-lower-estimate}
	Let $ f:[a,b] \to [0, \infty) $ be a continuous, piecewise-linear, convex function characterised by the singular points $ P = ( (x_i, y_i) )_{i \in [n]} $. Let $ l_{j} $ be the line joining points $ P_{j-1} $ and $ P_{j} $. Similarly, let $ l_{j+2} $ be the line joining points $ P_{j+1} $ and $ P_{j+2} $. Denote the intersection of the lines $ l_{j} $ and $ l_{j+2} $ by $ \bar{P} = ( \bar{x}, \bar{y} ) $.
	
	Then the function $ f_d: [a,b] \to [0, \infty) $ characterised by $ (P_0, \dots, P_{j-1}, \bar{P}, P_{j+2}, \dots, P_n) $ is also continuous, piecewise-linear and convex, and
	\begin{equation}
		f_d(x) \le f(x) \qquad \forall x \in [a,b]
	\end{equation}
\end{lmm}

\begin{proof}
	First we show the convexity of $f_d$. We know that $f$ satisfies the property of increasing slopes, that is $ m_{i} \le m_{i+1} \le m_{i+2} $. Since $f_d$ is obtained from $f$ by removing the line $l_{j+1}$, for $f_d$ we have that $ m_{i} \le m_{i+2} $, which implies that the function $f_d$ is still convex.
	
	Secondly, to prove the inequality, we may look at the convex function $f$ as if it has been obtained by removing point $ \bar{P} $ from the convex function $f_d$. Then, if $ \bar{x} \in ( x_{j} , x_{j+1} ) $, we have, using Lemma \ref{lmm:sp-asian-upper-estimate}, that $ f_d(x) \le f(x) \qquad \forall x \in [a,b] $.
\end{proof}

\begin{figure}
	\centering
	
	\definecolor{ffxfqq}{rgb}{1.,0.4980392156862745,0.}
	\definecolor{ffqqff}{rgb}{1.,0.,1.}
	\definecolor{cqcqcq}{rgb}{0.7529411764705882,0.7529411764705882,0.7529411764705882}
	\definecolor{yqyqyq}{rgb}{0.5019607843137255,0.5019607843137255,0.5019607843137255}
	\definecolor{zzttqq}{rgb}{0.6,0.2,0.}
	\definecolor{eqeqeq}{rgb}{0.8784313725490196,0.8784313725490196,0.8784313725490196}
	\begin{tikzpicture}[line cap=round,line join=round,>=triangle 45,x=0.8cm,y=0.8cm]
		\draw [color=eqeqeq,dotted, xstep=1.6cm,ystep=1.6cm] (0.,0.) grid (17.,13.);
		\draw[->,color=black] (0.,0.) -- (17.,0.);
		\foreach \x in {,2.,4.,6.,8.,10.,12.,14.,16.}
		\draw[shift={(\x,0)},color=black] (0pt,2pt) -- (0pt,-2pt);
		\draw[color=black] (16.672398939374787,0.08316246397131087) node [anchor=south west] { x};
		\draw[->,color=black] (0.,0.) -- (0.,13.);
		\foreach \y in {,2.,4.,6.,8.,10.,12.}
		\draw[shift={(0,\y)},color=black] (2pt,0pt) -- (-2pt,0pt);
		\draw[color=black] (0.10395306283371619,12.533153952113903) node [anchor=west] { y};
		\clip(0.,0.) rectangle (17.,13.);
		\draw [line width=1.2pt,color=yqyqyq] (8.,5.)-- (12.,8.);
		\draw [line width=1.2pt,color=yqyqyq] (16.,12.)-- (12.,8.);
		\draw [line width=1.2pt,color=yqyqyq] (8.,5.)-- (4.,3.);
		\draw [line width=1.2pt,color=yqyqyq] (4.,3.)-- (1.,2.);
		\draw [line width=0.4pt,color=cqcqcq] (12.,8.)-- (12.,0.);
		\draw [line width=0.4pt,color=cqcqcq] (8.,5.)-- (8.,0.);
		\draw [line width=0.4pt,color=cqcqcq] (4.,3.)-- (4.,0.);
		\draw [line width=0.4pt,color=cqcqcq] (1.,2.)-- (1.,0.);
		\draw [dash pattern=on 2pt off 2pt,color=ffqqff] (8.,5.)-- (10.,6.);
		\draw [dash pattern=on 2pt off 2pt,color=ffqqff] (12.,8.)-- (10.,6.);
		\draw [line width=0.4pt,color=cqcqcq] (10.,6.)-- (10.,0.);
		\draw [color=ffxfqq] (10.,6.)-- (10.,6.5);
		\draw [line width=0.4pt,color=cqcqcq] (16.,12.)-- (16.,0.);
		\begin{scriptsize}
			\draw [fill=zzttqq] (16.,12.) circle (1.5pt);
			\draw[color=zzttqq] (16.23579607547318,11.28571699254424) node {$P_5$};
			\draw [fill=zzttqq] (12.,8.) circle (1.5pt);
			\draw[color=zzttqq] (12.181626624958245,7.231546873942835) node {$P_4$};
			\draw [fill=zzttqq] (8.,5.) circle (1.5pt);
			\draw[color=zzttqq] (8.231410237277029,4.487185562889577) node {$P_3$};
			\draw [fill=zzttqq] (4.,3.) circle (1.5pt);
			\draw[color=zzttqq] (4.1980313993288405,2.4704958115852884) node {$P_2$};
			\draw [fill=zzttqq] (1.,2.) circle (1.5pt);
			\draw[color=zzttqq] (1.2041831897178144,1.6388711718721796) node {$P_1$};
			\draw [fill=zzttqq] (16.,0.) circle (1.5pt);
			\draw[color=zzttqq] (16.19421485033969,0.5369685242523106) node {$x_5$};
			\draw [fill=zzttqq] (12.,0.) circle (1.5pt);
			\draw[color=zzttqq] (12.140045399824759,0.5161779082594828) node {$x_4$};
			\draw [fill=zzttqq] (8.,0.) circle (1.5pt);
			\draw[color=zzttqq] (8.169038399576799,0.4953872922666551) node {$x_3$};
			\draw [fill=zzttqq] (4.,0.) circle (1.5pt);
			\draw[color=zzttqq] (4.1980313993288405,0.4953872922666551) node {$x_2$};
			\draw [fill=zzttqq] (1.,0.) circle (1.5pt);
			\draw[color=zzttqq] (1.1833925771510712,0.5577591402451383) node {$x_1$};
			\draw [fill=zzttqq] (10.,0.) circle (1.5pt);
			\draw[color=zzttqq] (10.22730904368438,0.5161779082594828) node {$x_{34}$};
			\draw [fill=ffqqff] (10.,6.) circle (1.5pt);
			\draw[color=ffqqff] (10.206518431117638,5.443553898559651) node {$Q_{34}$};
			\draw [fill=zzttqq] (10.,6.5) circle (1.5pt);
			\draw[color=zzttqq] (10.040193530583691,7.252337489935663) node {$P_{34}$};
			
		\end{scriptsize}
	\end{tikzpicture}
	
	\caption{Lower estimate: Illustration of Lemma \ref{lmm:sp-asian-lower-estimate} with $ j = 3, x_{34} = \bar{x}, P_{34} = \bar{P} $}
	\label{fig:lower-estimate}
\end{figure}


The lemmas \ref{lmm:sp-asian-upper-estimate} and \ref{lmm:sp-asian-lower-estimate}, will be used later to reduce the memory requirement of the algorithm by removing points or edges to simplify the function.



\section{Notations and conventions}
\label{sec:sp-asian-notations}

In this and subsequent sections, we shall use the convention that $ [n] = \{ 0, 1, 2, \dots, n \} $.

Let the number of time steps be $n$. Let $i$ denote the highlighted time step, and $j$ represent the number of up movements. In this way, we may represent any node by $ N_{i,j} $.

The price of the underlying at each node $ N_{i,j} $ is denoted by $ S_{i,j} $. Since there are $j$ up movements, there must be $ i-j $ down movements, and thus
\begin{equation} \label{eq:sp-asian-am-ij}
	S_{i,j} = S_0 u^{j} d^{i-j} = S_0 u^{j} u^{-(i-j)} = S_0 u^{-i+2j} \qquad \forall i \in [n], \ \forall j \in [i]
\end{equation}


\begin{prp}
	The number of paths to a node $ N_{i,j} $ is $ \binom{i}{j} $.
\end{prp}

\begin{proof}
	At each point in a path, we may choose either an up movement or a down movement. To reach node $ N_{i,j} $, we much choose $j$ up movements among $i$ possibilities. The result follows immediately.
\end{proof}


Any number of paths among the possible paths may give zero as the price for the option. We denote the number of singular points in a node $ N_{i,j} $ by $ L_{i,j} $, where $ L_{i,j} \in \left[ \binom{i}{j} \right] $. The $ l^\mathrm{th} $ average (in ascending order) ($ l \in \{ 1, \dots, L_{i,j} \} $) is denoted by $ A_{i,j}^l $, and the corresponding price by $ P_{i,j}^l $. Thus the singular points characterising the price function are $ ( ( A_{i,j}^l, P_{i,j}^l ) )_{l \in \{ 1, \dots, L_{i,j} \} } $.

\begin{dfn}[singular average and singular price]
	In the particular case of Asian options with arithmetic mean, the $ A_{i,j}^l $s are called `singular averages' and the $ P_{i,j}^l $s are called `singular prices'.
\end{dfn}


We recall some basic definitions and derive simple results for the maximum and minimum attainable value of the averages on each node.

Let the spot rate of interest be $r$ (constant) and the compounding be continous. Then, the effective compounding rate in each time period $\Delta t$ is given by $R$ as
\begin{equation}
	\label{eq:R}
	R = e^{r \Delta t}
\end{equation}
We note that the $R$ is not an instantaneous quantity, but one which is constant on an interval of time.


\begin{dfn}[Arithmetic mean]
	The arithmetic mean of a set of numbers $ \{ S_i \}_{i \in [n]} $ is given by:
	\begin{equation}
		\label{eq:am}
		A_{n} = \frac{\sum_{i=0}^n S_i}{n+1}
	\end{equation}
\end{dfn}


\begin{dfn}[Path]
	A path is a sequence $(j_i)_{i \in [n]}$ such that $j_{i+1} \in \{ j_i,j_i+1 \}$.
\end{dfn}

\begin{eg}
	See Figure \ref{fig:paths}.
\end{eg}


\begin{figure}
	% Recombining 4-step binomial tree for Cox-Ross-Rubinstein model
	\begin{tikzpicture}
		\matrix[column sep=10mm,row sep=1mm] (tree){
			& & & & \node[term] (u4) {$S_0u^4$}; \\
			& & & \node[nterm] (u3) {$S_0u^3$}; & \\
			& & \node[nterm] (u2) {$S_0u^2$}; & & \node[term] (u3d) {$S_0u^3d$}; \\
			& \node[nterm] (u) {$S_0u$}; & & \node[nterm] (u2d) {$S_0u^2d$};\\
			\node[term] (s) {$S_0$}; & & \node[nterm] (ud) {$S_0ud$}; & & \node[term] (u2d2) {$S_0u^2d^2$}; \\
			& \node[nterm] (d) {$S_0d$}; & &	\node[nterm] (ud2) {$S_0ud^2$};\\
			& & \node[nterm] (d2) {$S_0d^2$}; & & \node[term] (ud3) {$S_0ud^3$}; \\
			& & & \node[nterm] (d3) {$S_0d^3$}; & \\
			& & & & \node[term] (d4) {$S_0d^4$}; \\
		};
		% Lines out of s
		\draw[->,red,ultra thick] (s) -- (u) node[midway,above] {$p_u$};
		\draw[->,blue,thick] (s) -- (d) node[midway,below] {$p_d$};
		% Lines out of u
		\draw[->,red,ultra thick] (u) -- (u2) node[midway,above] {$p_u$};
		\draw[->,gray] (u) -- (ud) node[midway,above] {$p_d$};
		% Lines out of d
		\draw[->,blue,thick] (d) -- (ud) node[midway,below] {$p_u$};
		\draw[->,gray] (d) -- (d2) node[midway,below] {$p_d$};
		% Lines out of u2
		\draw[->,gray] (u2) -- (u3) node[midway,above] {$p_u$};
		\draw[->,red,ultra thick] (u2) -- (u2d) node[midway,above] {$p_d$};
		% Lines out of ud
		\draw[->,gray] (ud) -- (u2d) node[midway,above] {$p_u$};
		\draw[->,blue,thick] (ud) -- (ud2) node[midway,below] {$p_d$};
		% Lines out of d2
		\draw[->,gray] (d2) -- (ud2) node[midway,below] {$p_u$};
		\draw[->,gray] (d2) -- (d3) node[midway,below] {$p_d$};
		% Lines out of u3
		\draw[->,gray] (u3) -- (u4) node[midway,above] {$p_u$};
		\draw[->,gray] (u3) -- (u3d) node[midway,above] {$p_d$};
		% Lines out of u2d
		\draw[->,red,ultra thick] (u2d) -- (u3d) node[midway,above] {$p_u$};
		\draw[->,gray] (u2d) -- (u2d2) node[midway,above] {$p_d$};
		% Lines out of ud2
		\draw[->,gray] (ud2) -- (u2d2) node[midway,below] {$p_u$};
		\draw[->,blue,thick] (ud2) -- (ud3) node[midway,below] {$p_d$};
		% Lines out of d3
		\draw[->,gray] (d3) -- (ud3) node[midway,below] {$p_u$};
		\draw[->,gray] (d3) -- (d4) node[midway,below] {$p_d$};
	\end{tikzpicture}
	\caption{Two paths shown using red/thicker and blue/thick arrows. The other arrows are in grey/thin.}
	\label{fig:paths}
\end{figure}


\begin{thm}[Path inequality]
	\label{thm:sp-asian-up-dn-path}
	Let there be two paths $\alpha$ and $\beta$, such that $S_{i,j_i^\alpha} \ge S_{i,j_i^\beta} \; \forall i$, where $ ( j_i^\alpha )_{i \in [n]} $ and $ ( j_i^\beta )_{i \in [n]} $ denote the paths as defined above. Denote the corresponding averages by $A^\alpha$ and $A^\beta$, respectively. Then $ A^\alpha \ge A^\beta $.
\end{thm}

\begin{proof}
	Clearly if $S_{i,j_i^\alpha} = S_{i,j_i^\beta} \; \forall i$, then $A^\alpha = A^\beta$.
	
	We only need to show the result in the case of inequality.
	Let $ S_{i,j_i^\alpha} = S_{i,j_i^\beta} \; \forall i \in [n] \setminus \{l\} $, and $ S_{l,j_l^\alpha} > S_{l,j_l^\beta}$.
	
	Now, from equation \ref{eq:am}, we have:
	\begin{align*}
		(n+1) A_{n,j}^\alpha &= \sum_{i=0}^{l-1} S_{i,j_i} + S_{l,j_l^\alpha} + \sum_{i=l+1}^{n} S_{i,j_i} \\
		(n+1) A_{n,j}^\beta &= \sum_{i=0}^{l-1} S_{i,j_i} + S_{l,j_l^\beta} + \sum_{i=l+1}^{n} S_{i,j_i} \\
		\implies (n+1) \left(A_{n,j}^\alpha - A_{n,j}^\beta\right) &= S_{l,j_l^\alpha} - S_{l,j_l^\beta} \\
												 &= S_{l-1,j_{l-1}} u_l - S_{l-1,j_{l-1}} d_l \\
												 &= S_{l-1,j_{l-1}} (u_l - d_l) > 0 \qquad (u_l > d_l \text{ by definition}) \\
		\implies A_{n,j}^\alpha > A_{n,j}^\beta
	\end{align*}
\end{proof}


\begin{rem}
	The path $\alpha$ signifies a path `above' and $\beta$ a path `below' in the usual depiction of the binomial tree (the up movement shown above the down movement). Thus, any path above	 has a higher arithmetic mean than the one below.
\end{rem}


\begin{crr}
	\label{crr:sp-asian-up-dn-path}
	At each node $ N(i,j) $, the following hold:
	\begin{enumerate}
	\item The minimum average possible $ A_{i,j}^{\min} $ is attained by the path corresponding to the path corresponding to the path with $(i-j)$ down movements followed by $j$ up movements, and
		\begin{equation}	\label{eq:sp-asian-Amin}
			A_{i,j}^{\min} = \frac{S_0}{i+1} \left( \frac{1 - d^{i-j+1}}{1-d} + d^{i-j} u \frac{1 - u^{j}}{1-u} \right)
		\end{equation}
	\item The maximum average possible $ A_{i,j}^{\max} $ is attained by the path corresponding to the path with $j$ up movements followed by $(i-j)$ down movements, and
		\begin{equation} \label{eq:sp-asian-Amax}
			A_{i,j}^{\max} = \frac{S_0}{i+1} \left( \frac{1 - u^{j+1}}{1-u} + u^{j} d \frac{1 - d^{i-j-1}}{1-d} \right)
		\end{equation}
	\end{enumerate}
\end{crr}

\begin{proof}
	We show the proof only for the case of the maximum, since the case of the minimum can be shown using the exact same argument.
	
	From Theorem \ref{thm:sp-asian-up-dn-path}, the result about path with the maximum average holds directly, since there cannot be a path above the one given by $j$ up movements followed by $(i-j)$ down movements.
	
	The subsequent formula may be derived as follows.
	\begin{align*}
		(i+1) A_{i,j}^{\max} &= \underbrace{ ( S_0 + S_0 u + S_0 u^2 + \dots + S_0 u^j ) }_\text{up movement} + \underbrace{ ( S_0 u^j d + S_0 u^j d^2 + \dots + S_0 u^j d^{i-j} ) }_\text{down movement} \\
												 &= S_0 ( (1 + u + u^2 + \dots + u^j ) + u^j d ( 1 + d + \dots + d^{i-j-1} ) ) \\
												 &= S_0 \left( \sum_{k=0}^j u^k + u^j d \sum_{k=0}^{i-j-1} d^k \right) \\
												 &= S_0 \left( \frac{1 - u^{j+1}}{1-u} + u^{j} d \frac{1 - d^{i-j-1}}{1-d} \right) \qquad \text{(Geometric series)} \\
		\implies A_{i,j}^{\max} &= \frac{S_0}{i+1} \left( \frac{1 - u^{j+1}}{1-u} + u^{j} d \frac{1 - d^{i-j-1}}{1-d} \right)
	\end{align*}
\end{proof}

Table \ref{tab:sp-asian-notation} summarises the discussion above.

\begin{table}
	\label{tab:sp-asian-notations}
	\centering
	\caption{Summary of notations}
	\begin{tabular}{cccl}
		\toprule
		Symbol & Range & Formula & Description \\
		\midrule
		$ i $ & $ [ n ] $ & & highlighted time step \\
		$ j $ & $ [ i ] $ & & number of up movements \\
		$ N_{i,j} $ & & & node fixed by $ (i,j) $ \\
		$ S_{i,j} $ & $ [0, \infty) $ & Eq \ref{eq:sp-asian-am-ij} & value of the underlying at node $ N_{i,j} $ \\
		$ L_{i,j} $ & $ \left[ \binom{i}{j} \right] $ & & number of singular points in node $ N_{i,j} $ \\
		$ l $ & $ \{ 1, \dots, L \} $ & & index for points in ascending order of averages \\
		$ A_{i,j}^{\min} $ & $ [0, \infty) $ & Eq \ref{eq:sp-asian-Amin} & minimum average attainable for node $ N_{i,j} $ \\
		$ A_{i,j}^{\max} $ & $ [0, \infty) $ & Eq \ref{eq:sp-asian-Amax} & maximum average attainable for node $ N_{i,j} $ \\
		$ A_{i,j}^l $ & $ \left[ A_{i,j}^{\min}, A_{i,j}^{\max} \right] $ & Eq \ref{eq:am} & $ l^\mathrm{th} $ singular average of node $ N_{i,j} $ \\
		$ P_{i,j}^l $ & & & price corresponding to the average $ A_{i,j}^l $ \\
		$ (A_{i,j}^l, P_{i,j}^l) $ & & & $ l^\mathrm{th} $ singular point of node $ N_{i,j} $ \\
		\bottomrule
	\end{tabular}
\end{table}



\section{Fixed-strike European Asian options}
\label{sec:fixed-strike-eu}

For this type of options, the pay-off at maturity is dependent only on (some type of) average $ A_T $ at maturity $T$ and a fixed constant $K$, and is given by the function
\begin{equation} \label{eq:sp-asian-price-eu-asian-am}
	P_T = (A_T - K)_+
\end{equation}
We shall focus on this case in this section because it is the easiest to handle.

In each node of the binomial tree, we have a set of possible averages depending on the paths which may be taken to arrive at the node, and prices corresponding to each such average. We shall show these points to satisfy condition \ref{eq:sp-asian-conditions}, so they completely characterise a price function. Thus we focus not on the averages and corresponding prices possible under a particular binomial tree, but on the continuous representation of prices. The intuitive idea is that as the time step is reduced to zero, this function converges to the price function of the continuous time model.





\subsection{The price function at maturity}
\label{subsec:sp-asian-eu-price-maturity}

From equations \ref{eq:sp-asian-Amin} and \ref{eq:sp-asian-Amax}, putting $i = n$, we get
\begin{align*}
	A_{n,j}^{\min} &= \frac{S_0}{n+1} \left( \frac{1 - d^{n-j+1}}{1-d} + d^{n-j} u \frac{1 - u^{j}}{1-u} \right) \\
	A_{n,j}^{\max} &= \frac{S_0}{n+1} \left( \frac{1 - u^{j+1}}{1-u} + u^{j} d \frac{1 - d^{n-j-1}}{1-d} \right)
\end{align*}

In defining the price function, we note that three cases may arise.
\begin{itemize}
\item $ j \in \{ 0, n \} $ \\
	There exists only one path to these nodes, so there is only one average, implying one price and one singular point.
	
\item $ j \notin \{ 0, n \} $ and $ K \in ( A_{n,j}^{\min}, A_{n,j}^{\max} ) $ \\	
	The price function is characterised by three singular points ($ L_{i,j} = 3 $), $ ( A_{n,j}^l , P_{n,j}^l )_{l \in \{ 1, 2, 3 \} } $, which are \\
	\begin{equation} \label{eq:sp-asian-price-maturity-kin}
		\begin{aligned}
			( A_{n,j}^1 , P_{n,j}^1 ) &= ( A_{n,j}^{\min} , 0 ) \\
			( A_{n,j}^2 , P_{n,j}^2 ) &= ( K , 0 ) \\
			( A_{n,j}^3 , P_{n,j}^3 ) &= ( A_{n,j}^{\max} , A_{n,j}^{\max} - K ) \\
		\end{aligned}
	\end{equation} \label{eq:sp-asian-price-maturity-kout}
	
\item $ j \notin \{ 0, n \} $ and $ K \notin ( A_{n,j}^{\min}, A_{n,j}^{\max} ) $ \\
	The price function is characterised by only two singular points ($ L_{i,j} = 2 $), $ ( A_{n,j}^l , P_{n,j}^l )_{l \in \{ 1, 2 \} } $, which are \\
	\begin{equation}
		\begin{aligned}
			( A_{n,j}^1 , P_{n,j}^1 ) &= ( A_{n,j}^{\min} , ( A_{n,j}^{\min} - K )_+ ) \\
			( A_{n,j}^2 , P_{n,j}^2 ) &= ( A_{n,j}^{\max} , ( A_{n,j}^{\max} - K )_+ ) \\
		\end{aligned}
	\end{equation}
\end{itemize}

\begin{lmm}[Price function at maturity]
	\label{lmm:sp-asian-pr-maturity}
	At each node at maturity, the price function $ { v_{n,j}: \left[ A_{n,j}^{\min}, A_{n,j}^{\max} \right] \to \left[ ( A_{n,j}^{\min} - K )_+ , ( A_{n,j}^{\max} - K )_+ \right] } $ defined as $ v_{n,j}(A) = (A - K)_+ $ is continuous, piecewise-linear and convex.
\end{lmm}
\begin{proof}
	The singular points satisfy the conditions \ref{eq:sp-asian-conditions}. So for each $ A \in \left[ A_{n,j}^{\min}, A_{n,j}^{\max} \right] $, the price function ${ v_{n,j}(A) }$ characterised by the singular points is continuous, piecewise-linear and convex by remark \ref{rem:sp-asian-characterisation}.
\end{proof}



\subsection{The price function before maturity}
\label{subsec:sp-asian-eu-price-gen}

\begin{lmm}[Price function at any node]
	\label{lmm:sp-asian-dsc-expt}
	At any node $ N_{i,j} $, the price function $ v_{i,j}: \left[ A_{i,j}^{\min}, A_{i,j}^{\max} \right] \to [0, \infty) $ is continuous, piecewise-linear and convex.
\end{lmm}

\begin{proof}
	We shall prove this using backward induction, the base case at maturity being true by virtue of Lemma \ref{lmm:sp-asian-pr-maturity}.
	We now consider step $ i = n-1 $. Let $A_u$ and $A_d$ respectively represent the averages after an up and down movement corresponding to the average $A$. From equation \ref{eq:am}, we get
	\begin{subequations}
		\label{eq:sp-asian-av-up-dn}
		\begin{align}
			A_u &= \frac{ (i+1) A + S_0 u^{-i+2j+1} }{ i+1 }
						A_d &= \frac{ (i+1) A + S_0 u^{-i+2j-1} }{ i+1 }
		\end{align}
	\end{subequations}
	The price function $ v_{i,j}: \left[ A_{i,j}^{\min}, A_{i,j}^{\max} \right] \to [0, \infty) $ is obtained by considering the discounted expectation value.
	\begin{equation}
		\label{eq:sp-asian-dsc-expt}
		v_{i,j}(A) = \frac{1}{R} \left[ \pi v_{i+1,j+1}(A_u) + (1 - \pi) v_{i+1,j}(A_d) \right]
	\end{equation}
	From equation \ref{eq:sp-asian-av-up-dn}, we get that $A_u$ and $A_d$ are linear functions of $A$. Thus, $ v_{i+1,j+1}(A_u) = v_{n,j+1}(A_u)$ and $ v_{i+1,j}(A_d) = v_{n,j}(A_d) $ are piecewise-linear convex continuous functions of $A_u$ and $A_d$ respectively. Thus, $ v_{i+1,j+1} $ and $ v_{i+1,j} $ may be seen as composition of the above functions, and is thus piecewise-linear, convex and continuous itself. Again, from equation \ref{eq:sp-asian-dsc-expt}, we get that $v_{i,j}$ is a convex combination of such functions, and the proof is complete.
\end{proof}
TODO: Insert picture for this.


\begin{rem}
	From Lemma \ref{lmm:sp-asian-dsc-expt}, we see that the price function may be characterised by singular points.
\end{rem}



\subsection{Evaluation of singular points}
\label{subsec:sp-asian-eu-eval}

The evaluation of singular points for any node $ N_{i,j} $ is done by the following algorithm, which works in a backward fashion in time, starting from the maturity.

We note that for the only influencing nodes for the node $ N_{i,j} $ are $ N_{i+1,j+1} $ and $ N_{i+1,j} $. Thus we need to calculate the price of the option for each singular average belonging to either of these nodes.


\paragraph{Up movement}

First we take each singular average $ A_{i+1,j}^l $ belonging to $ N_{i+1,j} $ and project it to $ N_{i,j} $ via the relation
\begin{equation}
	\label{eq:sp-asian-proj-up}
	B^l = \frac{ ( i+2) A_{i+1,j}^l - S_0 u^{-i+2j-1} }{ i+1 }
\end{equation}
Thus, $ B^l $ is that average which after a down movement of the asset gives us the average $ A_{i+1,j}^l $.

Next, we note that $ B^l $ is an increasing function of $ l $, since a higher average at time step $ i $ would yield a higher average at time $ i+1 $. This in turn implies the following:
\begin{itemize}
\item $ B^{L_{i+1,j}} = A_{i+i,j}^{\max} \; \forall j $
\item $ B^1 \notin \left[ A_{i+i,j}^{\min}, A_{i+i,j}^{\max} \right] \ \forall j \in \{1, \dots, i-1 \} $
\end{itemize}
Each $ B^l \in \left[ A_{i,j}^{\min}, A_{i,j}^{\max} \right] $ becomes the singular average of $ N_{i,j} $.

In this way, we have determined the first coordinate of the singular points. We need to determine the second coordinate, or the prices $ v_{i,j}(B^l) $, $ \forall \left[ A_{i,j}^{\min}, A_{i,j}^{\max} \right] $, in order to determine the singular points completely. The idea is to calculate the discounted expected value of the price corresponding to each average $ B^l $ at $ N_{i,j} $. In order to be able to do this, we need the prices corresponding to the average projected to the node $ N_{i+1,j+1} $.

We consider an up movement of the underlying asset from node $ N_{i,j} $. In this case, $ B^l $ transforms into the average $ B^l_u = \left( (i+1) B^l + S_0 u^{-i+2j+1} \right) / ( i+2 ) $. Clearly, this average cannot belong to the set of averages associated with the node $ N_{i+1,j+1} $. Thus, we need to find the index $s$ such that $ B^l_u \in \left[ A_{i+1,j+1}^{s} , A_{i+1,j+1}^{s+1} \right] $. In the intervals the price function is linear, and thus we have
\begin{equation}
	\label{eq:sp-asian-up-lint}
	v_{i+1,j+1} \left( B^l_u \right) = \frac{ P_{i+1,j+1}^{s+1} - P_{i+1,j+1}^{s} }{ A_{i+1,j+1}^{s+1} - A_{i+1,j+1}^{s} } \left( B^l_u - A_{i+1,j+1}^{s} \right) + P_{i+1,j+1}^{s}
\end{equation}

We follow this up by calculating the price associated with the singular value $ B^l $ by evaluation the discounted expectation value.
\begin{equation}
	\label{eq:sp-asian-up-pr}
	v_{i,j}( B^l ) = \frac{1}{R} \left[ \pi v_{i+1,j+1} \left( B^l_u \right) + (1 - \pi) v_{i+1,j} \left( A_{i+1,j}^l \right) \right]
\end{equation}


\paragraph{Down movement}

We now proceed to formulate the theory for the downward movement in the exact same fashion. Define the new average $ C^l $ at the node $ N_{i,j} $ via the relation
\begin{equation}
	\label{eq:sp-asian-proj-dn}
	C^l = \frac{ ( i+2) A_{i+1,j+1}^l - S_0 u^{-i+2j+1} }{ i+1 }
\end{equation}

Again, we note that
\begin{itemize}
\item $ C^1 = A_{i,j}^{\min} \ \forall j $
\item $ C^{L_{i+1,j+1}} \notin \left[ A_{i,j}^{\min}, A_{i,j}^{\max} \right] \ \forall j \in \{1, \dots, i-1 \} $
\item $ C^l_d = \left( (i+1) C^l + S_0 u^{-i+2j-1} \right) / ( i+2 ) $
\end{itemize}
Each $ C^l \in \left[ A_{i,j}^{\min}, A_{i,j}^{\max} \right] $ becomes the singular average of $ N_{i,j} $.

For $ v_{i,j}( C^l ) $, $ \forall \left[ A_{i,j}^{\min}, A_{i,j}^{\max} \right] $, we now have the following.
\begin{equation}
	\label{eq:sp-asian-dn-lint}
	v_{i+1,j+1} \left( C^l_d \right) = \frac{ P_{i+1,j}^{s+1} - P_{i+1,j}^{s} }{ A_{i+1,j}^{s+1} - A_{i+1,j}^{s} } \left( C^L_d - A_{i+1,j}^{s} \right) + P_{i+1,j}^{s}
\end{equation}

\begin{equation}
	\label{eq:sp-asian-dn-pr}
	v_{i,j}( C^l ) = \frac{1}{R} \left[ \pi v_{i+1,j+1} \left( A_{i+1,j+1}^l \right) + (1 - \pi) v_{i+1,j} \left( C^l_d \right) \right]
\end{equation}


\paragraph{Aggregation}

Now we have the singular points for both up and down movements. We sort these points in ascending order of the first coordinate, i.e. the averages $ B^l $ and $ C^l $ that belong to $ \left[ A_{i,j}^{\min}, A_{i,j}^{\max} \right] $. These is an exhaustive list of all the singular points in the node (by construction). We note that $ L_{i,j} \le L_{i+1,j} + L_{i+1,j+1} - 2 $.

This procedure is applied to all nodes, starting from maturity and proceeding backwards. At the `edge' nodes $ N_{i,0} $ and $ N_{i,i} $, there is only one singular point whose price is given as follows
\begin{subequations}
	\label{eq:sp-asian-terminal-nodes}
	\begin{align}
		P_{i,0}^1 &= \frac{1}{R} \left[ \pi P_{i+1,0}^1 + (1 - \pi) P_{i+1,1}^1 \right] \\
		P_{i,i}^1 &= \frac{1}{R} \left[ \pi P_{i+1,i+1}^1 + (1 - \pi) P_{i+1,i}^{L_{i+1,i}} \right]
	\end{align}
\end{subequations}

Thus we have a complete description of the price function at each node of the binomial tree. The price $ P_{0,0}^1 $ is exactly the binomial price relative to the tree with $n$ steps of a fixed-strike European call option.


\section{Fixed-strike American Asian options}
\label{sec:fixed-strike-am}



TODO
\clearpage


\section{Extensions}
\label{sec:sp-asian-extensions}

Let us recapitulate the conditions required for the singular points method to work in the case of Asian options with arithmetic mean.
\begin{itemize}
\item The ability to calculate the upper and lower bounds of the mean for all nodes of the tree.
\item The recombinant nature of the tree for the underlying. Note that the tree for the option prices are \emph{not} recombinant.
\item Convexity and piecewise-linearity of the price function on the mean of the underlying.
\item Fixed volatility
\end{itemize}

Keeping these in mind, let us look at the possibility of extending the singular points method to the following cases:
\begin{enumerate}
\item Asian options with geometric mean and fixed volatility.
\item Asian options with arithmetic mean and local volatility.
\end{enumerate}



\subsection{Geometric mean and fixed volatility}
\label{subsec:gm-fixed-vol}

In the case of geometric options, we have a closed form formula under the Black-Scholes market model. We try to extend the singular points method.

Firstly, we show that the result about the maximum and minimum paths still hold in the geometric case.

\begin{dfn}[Geometric mean]
	The geometric mean of the risky asset's prices $ (S_i)_{i \in [n]} $ is given by:
	\begin{equation}
		\label{eq:gm}
		G_{n} = \left( \prod_{i=0}^n S_i \right) ^{\frac{1}{n+1}}
	\end{equation}
\end{dfn}


\begin{lmm}
	At each node $N(i,j)$, the following hold:
	\begin{enumerate}
	\item The maximum average possible $ G_{i,j}^{\max} $ is attained by the path corresponding to the path with $j$ up movements followed by $(i-j)$ down movements.
	\item The minimum average possible $ G_{i,j}^{\min} $ is attained by the path corresponding to the path corresponding to the path with $(i-j)$ down movements followed by $j$ up movements.
	\end{enumerate}
\end{lmm}

\begin{proof}
	The proof is the same as \ref{crr:sp-asian-up-dn-path}, with $A$ replaced by $G$ and relevant modifications.
\end{proof}


One of the central ideas behind the singular points method is that the price of the option is a convex, piecewise-linear function of the average $A$. But in the geometric case, this no longer holds true. For example, take a node $N_{i,j}$ with $ i = n-1 $. The price function given by $ v_{i,j}(G) $, with $ G \in [G^{min},G^{max}] $, can be calculated by the discounted expectation value.
\begin{align}
	v_{i,j}(G) &= \frac{1}{R} \left[ p v_{i+1,j+1}(G_u) + (1-p) v_{i+1,j}(G_d) \right] \\
	G_u &= \left( G^{i+1} S_0 u^{-i+2j+1} \right)^{\frac{1}{i+2}} \propto G^{\frac{i+1}{i+2}} \\
	G_d &= \left( G^{i+1} S_0 u^{-i+2j-1} \right)^{\frac{1}{i+2}} \propto G^{\frac{i+1}{i+2}}
\end{align}
Clearly, the final function $ v_{i,j} $ is not linear in $G$. Rather it is piecewise-concave. Thus we cannot use the singular points method in this case.
TODO: Insert a graph of the function here.



\subsection{Arithmetic mean with local volatility}
\label{subsec:am-local-vol}

In this case, the tree for the underlying is not recombinant, so we do not have more than one singular point in one (non-recombining) node. Clearly, we cannot use the singular points method.



\section{Conclusion}
\label{sec:sp-asian-adv}

We conclude the chapter by noting the pros and cons of the singular points method.
\paragraph{Advantages}
\begin{itemize}
\item Fast -- Experimental order of complexity = $ O(n^3) $
\item It allows us to specify an \emph{a priori} error bound.
\end{itemize}


\paragraph{Disadvantages}
\begin{itemize}
\item Very specific method -- only applicable to a few specific cases.
\end{itemize}


%%% Local Variables:
%%% mode: latex
%%% TeX-master: t
%%% End:
