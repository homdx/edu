% !TeX root = thesis.tex
% !TeX spellcheck = en_GB
% !TeX encoding = UTF-8


\documentclass[12pt,a4paper]{amsbook}

\usepackage{thesis}


\begin{document}


\frontmatter

% Half-title

% Title page
% !TeX root = thesis.tex
% !TeX spellcheck = en_GB
% !TeX encoding = UTF-8


\begin{titlepage}
	\begin{center}

		% Logos
		\includegraphics[width=\textwidth]{logos.jpg} \\[3em]

		% Programme information
		{\Large\bfseries Erasmus Mundus Consortium} \\[1em]
		{\LARGE\bfseries MathMods} \\[2em]
		
		% Joint degree
		{\large Joint Degree of Master of Science in} \\
		{\large Mathematical Modelling in Engineering: Theory, Numerics, Applications} \\[1em]
		
		{\large In the framework of the} \\
		{\large Consortium Agreement and Award of a Joint/Multiple Degree 2013-2019} \\[3em]
		
		% Title
		{\LARGE Master's Thesis} \\[2em]
%		\hrule
		{\Huge\bfseries The Singular Points method for pricing exotic path-dependent options}
%		\hrule \leavevmode
		\\[4em]

		% Supervisor
		\begin{minipage}[t]{0.5\textwidth}
			\begin{center} \Large
				\emph{Supervisor:} \\
				\textsc{\LARGE Prof. \textbf{Fabio Antonelli}}
			\end{center}
		\end{minipage}
		\begin{minipage}[t]{0.4\textwidth}
			\begin{center} \Large
				\emph{Candidate:}\\
				{\LARGE\textbf{\textsc{Sudip Sinha}}}\\
				Matricola: 228435
			\end{center}
		\end{minipage}
		
		\vfill

		% Bottom of the page
		{\large \today}

	\end{center}
\end{titlepage}


%%% Local Variables:
%%% mode: latex
%%% TeX-master: t
%%% End:


%A path-dependent option is a type of exotic option in which the price of the option depends not only on the price of the underlying asset at maturity, but also on the history of the asset's price till that point. Some typical examples of popular exotic options are Asian options, lookback options, barrier option, digital options, etc.
%
%In the Black-Scholes market model, it is not possible to find closed-form analytical formulae for the prices of many exotic options. Thus we need fast numerical algorithms to determine the fair price of such options. Such algorithms may be clustered into groups. Some algorithms utilise the converence of the Cox-Ross-Rubinstein prices to the Black-Scholes prices as a limit when the time step is reduced. Other algorithms use numerical methods to solve the Black-Scholes stochastic differential equation. Other use Monte Carlo methods...
%
%Under the Cox-Ross-Rubinstein model, Gaudenzi \emph{et al}\cite{Gaudenzi2010} introduced the Singular Points method. It is a numerical method to price American Lookback options and American Asian options with arithmetic mean. In each node of the binomial tree of the underlying risky asset, the price is represented as a continuous function of the path-dependent parameter. An advantage of this method over pre-existing methods is its low order of computational and space complexity. It is convergent, and allows us to set \emph{a priori} upper and lower bounds on the error.
%
%In the master's thesis, we try to extend the method to include the case of American Asian options with geometric mean. I would also like to look into the possibility of extending the method to account for local volatility.

%Research is what I’m doing when I don’t know what I’m doing. — Wernher von Braun

% Information (copyright notice, ISBN, etc.)

% Dedication if any, else empty

\tableofcontents

%\clearpage


\section*{Notations}

\subsection*{Mathematical}

\begin{description}
\item[$ {[n]} $] $ \{0, 1, 2, \dots, n\} $
\item[$ \mathbb{N} $] $ \{ 1, 2, 3, \dots \} $
\item[$ \mathbb{N}_0 $] $ \mathbb{N} \cup \{ 0 \} $
\item[$ \mu \mathcal{F} $] The class of all $ \mathcal{F} $-measurable functions.
\end{description}

%\subsection*{General}
%\begin{description}
%\item[CRR] The Cox-Ross-Rubinstein model
%\item[BS] The Black-Scholes model
%\end{description}
%
%Mention the assumptions - knowledge on basic Probability Theory and Stochastic Processes. Probability space, Filtrations,RVs, Stochastic processes, 
%No need to know about Finance.

% Preface chapter


\mainmatter

\chapter{Prologue}
\label{cha:prologue}
% !TeX root = ../thesis.tex
% !TeX spellcheck = en_GB
% !TeX encoding = UTF-8


[TODO] One page history
\begin{itemize}
	\item The thesis of Louis Bachelier (1900) on the ``Theory of Speculation''
	\item Introduction of Brownian motion to model fluctuating prices in the Paris stock exchange
	\item Black and Scholes
	\item Cox Ross Rubinstein
\end{itemize}



\section{Financial instruments}
\label{sec:intro-assets}

A \emph{financial instrument} or a \emph{financial asset} is an intangible asset whose value is derived from a contractual claim, such as bank deposits, bonds, stocks and derivatives. Financial assets are usually more liquid than other tangible assets, such as commodities or real estate, and may be traded on financial markets. Every financial asset is characterised by its return. When the return is deterministic, we call it a \emph{risk-free} or \emph{riskless} asset. When the return is contingent on the market and external conditions, it is called \emph{risky}. It must be kept in mind that no instrument is fundamentally risk-free, it has only negligible risk compared to its risky counterparts.



\subsection{Return on an asset}
\label{subsec:intro-assets-return}

\paragraph{Return on a riskless asset -- Compounding}
Compounding is the first idea that we must be familiar with. Essentially, a riskless asset will increase in monetary value in a deterministic manner if we keep it in the market. The increase depends on the compounding frequency and the duration of investment. The term compounding is used because the interest earned in each period also contributes to the principal in the successive periods.

Let the compounding frequency is $ n $ times per year, the total time is $ t $, and the annual rate of interest is $ r $. Then
\begin{equation}
	\label{eq:intro-compounding-discrete}
	B_t = B_0 \left(1 + \frac{r}{n} \right)^{\floor{nt}},
\end{equation}
where $ B_0 $ is the starting value of the asset, and $ B_t $ is the value of the asset at time $ t $, and $ \floor{\cdot} $ represents the floor function.

If the compounding is continuous, we let $ n \to \infty $ to obtain
\begin{equation}
	\label{eq:intro-compounding-continous}
	B_t = B_0 e^{rt}.
\end{equation}


\paragraph{Return on a risky asset}
At any point of time in the future, the value of a risky asset is not known with certainty, so it is a random variable taking values in $ [0, \infty) $. Since the values change with time, we denote the \emph{spot price} of a risky asset by the stochastic process $ (S_t)_t $, where $ t \in [0, T] $ denotes the time and $ T $ is the maturity. Since the future value of the asset is adventitious, we use the following quantities to measure the return of the risky asset in the given time interval.

\begin{dfn}[absolute and relative returns]
	The absolute return on an asset for the time interval $ [0, t], \  t \in [0, T] $ is given by
	\begin{equation*}
		\tilde{R}_t = S_t - S_0
	\end{equation*}
	
	The relative return on the asset is given by
	\begin{equation*}
		R_t = \frac{S_t - S_0}{S_0}
	\end{equation*}	
\end{dfn}



\subsection{Common asset types}

\paragraph{Bonds}
A \emph{bond} is an instrument of indebtedness of the bond issuer to the holders. It is a \emph{debt security}, under which the issuer owes the holders a debt and, depending on the terms of the bond, is obliged to pay them interest (the coupon) and/or to repay the principal at a later date, termed the maturity date. Bonds can also be thought of as a \emph{loan} given to the issuer of the bond by the holder. A bond primarily has two kinds of risks, \emph{credit default risk} and \emph{interest rate risk}. A bond issued by a reliable institution like the United States government is a good illustration of a risk-free asset. This is because the probability of such an organisation defaulting is close to zero, or in other words, the bond has \emph{negligible} \emph{credit default risk}. Such bonds are only subject to fluctuations of the current interest rate, called \emph{interest rate risk}. The interest rate risk may also be nullified if the bond is held till maturity. If we assume that the interest rate is deterministic (the fluctuations are not random), the value of the bond is computable at any given future date, making it riskless. Such an assumption is quite reasonable in short periods of time and for institutions with a low default risk rating.


\paragraph{Stocks}
A \emph{stock} of a corporation is an ownership certificate, and constitutes the equity stake of its owners. It represents the residual assets of the company that would be due to stockholders after discharge of all senior claims such as secured and unsecured debt. A \emph{share} of a stock is a unit of ownership of the organisation. Stocks are inherently risky, since the value of the organisation may change from time to time due to various internal and external factors. The value of a stock in time is usually represented by a stochastic process $ (S_t)_t $.


\paragraph{Derivatives}
A \emph{derivative} is a contract between two parties that specify conditions (starting and termination dates, resulting values and definitions of the underlying variables, the parties' contractual obligations, and the notional amount) under which transactions are to be made between the parties. The most common underlying assets include commodities, stocks, bonds, interest rates and currencies, but they can also be other derivatives, which adds another layer of complexity to proper valuation. Essentially, the value of a derivative is a function of the value of the \emph{underlying} asset(s). Derivatives are traded in their own right, and a fair price must be found for a derivative at each time of its existence. This problem is known as the \emph{pricing problem}. One of the primary motivations for creation of derivatives was to hedge one's position from fluctuations in the market. A \emph{hedge} is an investment strategy intended to offset potential losses or gains that may be incurred by a companion investment. Finding a hedging strategy is called the \emph{hedging problem}. These are the two problems that must be looked at when defining a market model. In this thesis, our main focus shall be the pricing problem of a particular class of derivatives, called \emph{exotic options}.


\subsection{Classification of derivatives}
\label{subsec:intro-derivative-classification}

Derivatives may be classified on the basis of various factors. One important factor is whether the risk is shared, or taken up by only one party. Another factor is the nature of the function (of the underlying) that the derivative depends on. This function may either be dependent only on the final value of the underlying (\emph{path-independent}), or on the path that it took to reach this final value (\emph{path-dependent}). The function may be discrete (\emph{digital} or \emph{binary}), or continuous In this section, we briefly look at some of the more important types of derivatives. \footnote{
	A more interested reader should consult the following extensive Wikipedia articles.
	\begin{itemize}
		\item \url{https://en.wikipedia.org/wiki/Option_(finance)\#Types}
		\item \url{https://en.wikipedia.org/wiki/Option_style}
	\end{itemize}
}


\paragraph{Futures and forwards}

\begin{dfn}[Futures and forwards]
	Futures and forwards are contracts between two parties, the seller and the buyer, to exchange a certain asset at a predetermined future time at a agreed upon price. Futures are \emph{exchange-traded derivatives} (ETDs), whereas forwards are traded \emph{over-the-counter} (OTC).
\end{dfn}

Such derivatives obligate the contractual parties to the terms over the life of the contract. Futures are in some sense `safer' compared to forwards, since the involved parties must go through standard protocols of the exchange.
The contract contains the following details.
\begin{description}
	\item[$ T $] The maturity, or the duration of the contract
	\item[$ F_0 $] The delivery price, or the price prefixed (at the initial time) at which trades must take place at maturity
	\item[$ r $] The rate of interest
	\item[underlying] The asset(s) of trade at maturity
	\item[$ S_0 $] The initial value of the underlying asset(s)
\end{description}
There are, of course, other possibilities, for instance a variable interest rate, dividends yielded by the underlying, but these may be viewed as generalisations of this simple case.

Let us assume that the compounding is continuous. We may show that under the condition of a \emph{viable market}\footnote{see Section \ref{sec:intro-market} for definitions of the term}, the fair delivery price of a future with underlying prices $ ( S_t )_{t \in [0, T] } $ at any time $ t \in [0, T] $ is given by the following equation.
\begin{equation}
	\label{eq:intro-future-pr}
	F_t = S_t e^{ r (T - t) }
\end{equation}


\paragraph{Options}

\begin{dfn}[option]
	An \emph{option} or a \emph{contingent claim} is a derivative which provides the buyer \emph{the right, but not the obligation} to enter the contract under the specified terms.
\end{dfn}

Thus, the owner of the option may choose whether to exercise his right or not. Thus, on the one hand, the owner of the option bears no risk, since all the choice is his. On the other hand, the seller of the option is \emph{obligated} to honour the terms of the contract -- whether it benefits him or not -- essentially making him bear all the risks. This asymmetry is primarily what sets options apart from the futures and forwards discussed earlier.
The contract contains the following details.
\begin{description}
	\item[$ T $] The maturity, or the duration of the contract
	\item[$ K $] The strike price, or the prefixed price at which trades may take place at maturity
	\item[$ r $] The rate of interest
	\item[underlying] The asset(s) which may be traded at maturity
	\item[$ S_0 $] The initial value of the underlying asset at the initial time
	\item[right] The exact right that the owner of the options has (see below)
	\item[exercise time] European or American (see below)
\end{description}

Options may be categorised by the right of the owner and the exercise time.

\subparagraph{According to the right of the owner}  Options may be of two main types.
\begin{description}
	\item[call] The owner has the right to buy. In this case, the price of the option at maturity is given by $ c_T = (S_T - K)_+ $, where $ (x)_+ \coloneqq \max \{ 0, x \} $.
	\item[put] The owner has the right to sell. In this case, the price of the option at maturity is given by $ p_T = (K - S_T)_+ $.
\end{description}
Of course, other complicated ownership rights may be constructed, but we shall not go into the details of those.

\subparagraph{According to the time at which the option may be exercised}  Again, option may be of two main types.
\begin{description}
	\item[European] The owner may exercise the option only at maturity
	\item[American] The owner may exercise the option at any time up to the maturity
\end{description}
Again, more complicated options exist, which allow exercising rights only at certain time points, but we exclude them from out discussion.

Since American options allow for more flexibility for the owner, and thus more risk for the seller, they are more expensive as compared to their European counterparts. Let $ c_t, p_t $ denote the prices of an European call and put, and $ C_t, P_t $ denote the prices of an American call and put, respectively. Then, we must have $ C_t \ge c_t $ and $ P_t \ge p_t $.


\subparagraph{Call-put parity}  Call and put prices are connected to each other. We need the following proposition to explore the relationship.

\begin{prp}[Equality of portfolios]
	\label{thm:intro-portfolio-eq}
	In a \emph{viable} and \emph{frictionless market}\footnote{see Section \ref{sec:intro-market} for definitions of the terms}, if the values of two portfolios coincide at a time $ T $, they have to coincide at $ 0 $ ( or any other intermediate time $ t $).
\end{prp}

\begin{proof}
	Let us denote by $ \mathcal{P}_1 $ and $ \mathcal{P}_2 $ the two portfolios and by $ v(\mathcal{P}) $ the value of a portfolio $ \mathcal{P} $ at time $ t $. By assumption $ v_T (\mathcal{P}_1) = v_T (\mathcal{P}_2) $, so we assume by contradiction that $ v_T (\mathcal{P}_1) > v_T (\mathcal{P}_2) $.
	
	Under this hypothesis it is possible to construct the following arbitrage strategy. At time 0, one can borrow the portfolio $ \mathcal{P}_1 $ and sell it right away to buy portfolio $ \mathcal{P}_2 $. One can pocket the difference $ v_T (\mathcal{P}_1) - v_T (\mathcal{P}_2) > 0 $. At $ t = T $ the values of the two portfolios coincide, so selling $ \mathcal{P}_2 $ one gets the exact money to buy $ \mathcal{P}_1 $ to be returned to the original lender. An profit is achieved, without investing any money, implying an arbitrage and violating the viable market hypothesis. Similarly, we can show that $ v_0 (\mathcal{P}_1) < v_0 (\mathcal{P}_2) $ would also enable an arbitrage opportunity.
\end{proof}

We now look at the relationship. $ S_T - K = (S_T - K)_+ + (S_T - K)_- = (S_T - K)_+ - (K - S_T)_+ = c_T - p_T $. For any general time $ t $, using Proposition \ref{thm:intro-portfolio-eq}, it holds that $ c_t - p_t = S_t - K e^{- r (T-t) } $. This is known as the \emph{call-put parity}.


\paragraph{Exotic options}
European options are path-independent and the simplest type of options available. Hence, they are popularly known as \emph{vanilla options}. The American options are path-dependent. Typically, other options which are more complex in nature are collectively called \emph{exotic options}. These are usually path-dependent, and may be either European, American or have more complex exercise times. A few such options are described in brief.
\begin{description}
	\item[Asian] The payoff depends on the average of the underlying's prices
	\item[lookback] The payoff depends on one of the extrema of the underlying's prices
	\item[cliquet or ratchet] A series of globally or locally, capped or floored, at-the-money options, but where the total premium is determined in advance.
	\item[barrier] The price of the underlying reaching the pre-set barrier level either springs the option into existence (\emph{knock-in}) or extinguishes an already existing option (\emph{knock-out}).
\end{description}



\section{Financial Markets}
\label{sec:intro-market}

The idea of financial markets is intricately linked to that of financial transactions. Analogous to the ordinary markets, a financial market is a human construct to allow transaction between investors. The assets in the financial market are typically financial instruments such as bonds, stocks and derivatives discussed in the previous section. In this section we will primarily concern ourselves with the nature of financial markets and the assumptions we make while modelling them. Some of the jargon used in the previous section will become clear after this section.

Pricing of financial assets is one of the most pressing aims of the subject of Financial Mathematics. In order to do so, we need to understand and characterise the fundamental mechanisms of the market that shape the pricing of assets. In doing so, we must note which dynamics of the market are most fundamental and must be incorporated in every model, and which are more debatable may be excluded from simpler models.

\paragraph{Viable market}
The term viability here refers to the fairness of a market. To interpret viability, we need to familiarise ourselves with the following definitions.

In what follows, we assume the following.
\begin{itemize}
	\item There is a probability space $ (\Omega, \mathcal{F}, (\mathcal{F}_t)_t, P) $, endowed with the filtration $ (\mathcal{F}_t)_t $.
	\item $ \forall t \in [0, T], T \in [0, \infty) $, there is one riskless asset worth $ S_t^0 = e^{rt} $ (Take $ S_0^0 = 1 $).
	\item $ \forall t \in [0, T], T \in [0, \infty) $, there are $ d $ risky assets each worth $ S_t^i $, where $ i \in \{ 1, 2, \dots, d \} $ is the index of the risky asset. These may be represented as a $ d $-dimensional (vector) stochastic process $ ( S_t^1, S_t^2, \dots, S_t^d ) $.
\end{itemize}

Also, any quantity with a $ \sim $ on top denotes the discounted value of that quantity.


\begin{dfn}[investment strategy]
	A $ d + 1 $ dimensional (vector) stochastic process $ \Phi = (\bm{\phi_t})_t = (\phi_t^0, \phi_t^1, \dots, \phi_t^d)_t $ is called an \emph{investment strategy} or \emph{trading strategy} if $ \phi_t^i $ is $ \mathcal{F}_t $-measurable $ \forall i \in [d] $.
\end{dfn}
This means that there is a procedure to allocate resources within the portfolio at all times. We shall write strategy to mean investment strategy in the rest of the document.

The following definition gives the value of a strategy at a point in time.
\begin{dfn}[value of a strategy]
	The value of a strategy $ \Phi $ at a time $ t $ is given by $ V_t( \Phi ) = \bm{\phi_t} \cdot S_t $.
\end{dfn}

\begin{dfn}[self-financing strategy]
	A portfolio is self-financing if its changes in value are only due to changes in prices of the assets. This can be represented as follows.
	\begin{subequations}
		\begin{align}
			\dif V_t &= \bm{\phi_t} \cdot \dif S_t \qquad \forall t \in [0, T]  \\
			\implies  \qquad  \dif \tilde{V}_t &= \bm{\phi_t} \cdot \dif \tilde{S}_t \qquad \forall t \in [0, T]
		\end{align}
	\end{subequations}
\end{dfn}
This implies that we do not put in any fresh money in the strategy at any point of time, apart from what is generated due to the change is values of the underlying assets.

\begin{dfn}[admissible strategy]
	A self-financing strategy $ \Phi $ is said to be admissible if $ V_t( \Phi ) \ge 0 \  \forall t \in [0, T] $.
\end{dfn}
This implies that we do not run out of money at any point of time.

\begin{dfn}[arbitrage strategy]
	An admissible strategy $ \Phi $ is said to be an arbitrage strategy if $ V_0( \Phi ) = 0 $ and $ P( V_T( \Phi ) > 0 ) > 0 $.
\end{dfn}
A arbitrage strategy basically means that we generate value at time $ T $ without any initial investment. For the sake of fairness, we do not want a market in which there exist arbitrage opportunities. The next definition addresses this issue.

%In what follows, we assume the following.
%\begin{itemize}
%	\item There is a probability space $ (\Omega, \mathcal{F}, (\mathcal{F}_n)_n, P) $, endowed with the filtration $ (\mathcal{F}_n)_n $.
%	\item $ \forall n \in [N], N \in \mathbb{N} $, there is one riskless asset worth $ S_n^0 = e^{rt} $ ($ S_0^0 = 1 $).
%	\item $ \forall t \in [0, T], T \in [0, \infty) $, there are $ d $ risky assets each worth $ S_t^i $, where $ i \in \{ 1, 2, \dots, d \} $ is the index of the risky asset. These may be represented as a $ d $-dimensional (vector) stochastic process $ ( S_t^1, S_t^2, \dots, S_t^d ) $.
%\end{itemize}
%
%A word about notation: We shall denote the class of all $ \mathcal{F} $-measurable random variables by $ \mu \mathcal{F} $.
%
%\begin{dfn}[investment strategy]
%	A $ d + 1 $ dimensional (vector) stochastic process $ \Phi = (\bm{\phi_t})_t = (\phi_t^0, \phi_t^1, \dots, \phi_t^d)_t $ is called an \emph{investment strategy} or \emph{trading strategy} if $ \phi_t^i \in \mu \mathcal{F}_t \  \forall i \in [d]$.
%\end{dfn}
%This means that there is a procedure to allocate resources within the portfolio at all times. We shall write strategy to mean investment strategy in the rest of the document.
%
%The following definition gives the value of a strategy at a point in time.
%\begin{dfn}[value of a strategy]
%	The value of a strategy $ \Phi $ at a time $ t $ is given by $ V_t( \Phi ) = \bm{\phi_t} \cdot S_t $.
%\end{dfn}
%
%\begin{dfn}[self-financing strategy]
%	A strategy $ \Phi $ is called \emph{self-financing} if $ \dif \bm{\phi_t} \cdot S_t = 0 \  \forall t \in [0, T] $.
%\end{dfn}
%This implies that we do not put in any fresh money in the strategy at any point of time, apart from what is generated due to the change is values of the underlying assets.
%
%\begin{dfn}[admissible strategy]
%	A self-financing strategy $ \Phi $ is said to be admissible if $ V_t( \Phi ) \ge 0 \  \forall t \in [0, T] $.
%\end{dfn}
%This implies that we do not run out of money at any point of time.
%
%\begin{dfn}[arbitrage strategy]
%	An admissible strategy $ \Phi $ is said to be an arbitrage strategy if $ V_0( \Phi ) = 0 $ and $ P( V_T( \Phi ) > 0 ) > 0 $.
%\end{dfn}
%A arbitrage strategy basically means that we generate value at time $ T $ without any initial investment. For the sake of fairness, we do not want a market in which there exist arbitrage opportunities. The next definition addresses this issue.
%
%\begin{prp}[Equivalence of arbitrage conditions]
%	Assume that $ \phi_t^i \in L^2( P \times \lambda ) $, where $ \lambda $ is the Lebesgue measure for time. Then the following are equivalent.
%	\begin{itemize}
%		\item $ \Phi $ is self-financing
%		\item $ \forall t \in [0, T], \dif V_t = \bm{\phi_t} \cdot \dif S_t $
%		\item $ \forall t \in [0, T], \dif \tilde{V}_t = \bm{\phi_t} \cdot \dif \tilde{S}_t $
%	\end{itemize}
%\end{prp}
%[TODO: Proof]
%[TODO: May be incorrect -- Check \cite[Remark 4.1.3, Page 65]{Lamberton1996}]


\begin{dfn}[viable market, no free lunch]
	A market is called \emph{viable}, or there is \emph{no free lunch}, if there does not exist any arbitrage strategies.
\end{dfn}

We will see in Theorem \ref{thm:discrete-ftoap1} of Chapter \ref{cha:models} how the financial concept of viability translates mathematically to the existence of equivalent martingale measures.


\paragraph{Complete market}

\begin{dfn}[hedgeable, attainable]
	A derivative is called hedgeable or attainable if it can be represented as a linear combination of its underlying assets at all times.
\end{dfn}

\begin{dfn}[complete market]
	A market is said to be complete if all derivatives are hedgeable.
\end{dfn}

A market is complete if the traded basic assets represent all the random factors that influence the course of prices. If we are in complete market model, this means that, whatever be the contract, one can always set a hedging strategy that equals the final value of the derivative. This hedging
strategy employs, by definition, only the basic assets (riskless and risky) traded on the market. In a sense those derivatives are redundant, they do not introduce any additional risk factor; any randomness source is represented by the basic assets prices and it is tradeable.

This implies that the marginal probability distributions of the derivative price are uniquely determined by the marginal distributions of the prices of the basic assets. Namely, each derivative is replicable, so its price at all times may be written as the value of a portfolio employing basic assets only, that is to say it is a linear combination of the prices of the basic assets. On the other hand, in a viable market the discounted value of a portfolio is a martingale under the risk-neutral probability, so if a contingent claim is attainable, then its price, by the martingale property, belongs to the linear span of the basic assets' prices under this probability. Vice versa, if the market is incomplete, then there must be sources of randomness that cannot be totally represented as linear combinations of the prices of the basic tradeable assets, which means that the basic titles are not sufficient to construct all the necessary hedging strategies.

We will see in Theorem \ref{thm:discrete-ftoap2} of Chapter \ref{cha:models} how the financial concept of market completeness translates mathematically to the existence of a unique equivalent martingale measure.


\paragraph{Frictionless market}
For any transaction (sale or purchase) in the market, one usually pays some \emph{commission}. The commission is a very small fraction of the current value of the traded assets, and it seems reasonable to assume that it is not a factor that affects the dynamics of the prices in a direct fashion. Furthermore, the computational difficulty of including such transactional costs in the market is quite high. Thus, we choose to ignore such costs in the simple market models that we shall deal with, and call the market as \emph{frictionless}.

\begin{dfn}[frictionless market]
	A market is called frictionless if there are no transaction costs.
\end{dfn}


\paragraph{Infinitely divisible assets}
In a market, usually only discrete units of assets may be bought or sold. This would pose an additional constraint in the modelling of the market. But it is quite evident that this constraint does in no way affect prices of individual assets. Furthermore, markets are usually so varied that one might think to combine stocks and bonds to a value that is roughly equivalent to a fraction of a different asset. Thus, we ignore this constraint in our market models, and say that we may have \emph{infinitely divisible assets} in our market.


\paragraph{Small investor hypothesis}
An investor who has virtually unlimited funds might decide to buy massive quantities of an asset to make its price rise in order to sell it later at a higher price. We shall ignore such cases, and assume that all agents are trifling with respect to the market dimension, meaning that they cannot influence prices uniquely by means of their investing strategies, hence prices are determined only by the combined actions of all agents. This assumption is called the \emph{small investor hypothesis} and it is totally realistic in bigger stock markets like for those in the United States, even though it is less so for much smaller markets.


\paragraph{Borrowing}
Lastly, we assume that an investor may borrow assets, whether they are financial instruments such as money, bonds and stocks, or otherwise.


%%% Local Variables:
%%% mode: latex
%%% TeX-master: t
%%% End:


\chapter{Cox Ross Rubinstein model}
\label{cha:crr}
\input{tex/crr.tex}

\chapter{Black Scholes model}
\label{cha:bs}
\input{tex/bs.tex}

\chapter[Asian options]{The Singular Points method applied to Asian options}
\label{cha:sp-asian}
% !TeX root = ../thesis.tex
% !TeX spellcheck = en_GB
% !TeX encoding = UTF-8


As we have seen in the earlier chapters, European options may be priced using the Cox-Ross-Rubinstein and the Black-Scholes models. Even though the Black-Scholes model has a very high degree of computability, it does not allow us to find closed-form pricing formulae for many path-dependent options, including American options. The way out is by using numerical methods. Numerical methods using discrete models were introduced in Chapter \ref{cha:models}. One simple idea is to apply an explicit pricing scheme using a lattice method, which converges to the Black-Scholes model as the number of time steps increases to infinity. But the exponential number of paths ($2^n$ to be exact, where $n$ is the number of time steps) make the method very slow and memory intensive, making it computably impractical. A logical step would be to modify the basic Cox-Ross-Rubinstein model to allow for approximations. In this direction, Gaudenzi et al\cite{Gaudenzi2010} introduced a new method called the \emph{singular points method} for pricing certain path-dependent options in an efficient manner. The chapter is a study on how this method and its applications.

We will mainly focus using the singular points method for pricing Asian options. In Asian options, the price is expressed as a function of some form of averaging on the underlying's price. Popular Asian options use the arithmetic or geometric means as the average. Again, Asian options may be exercised only at maturity (European) or at any time till the maturity (American). They may give the owner of the option the right to either sell (put) or buy (call). Theoretically, we will only study calls, because the framework for puts one may be derived in the exact same way.



\section{Literature Review}
\label{sec:asian-literature-review}

Before we go into the details of the singular points method, we shall look into the pre-existing methods of pricing Asian options, and discuss their advantages and disadvantages briefly. As we remarked in Section \ref{subsec:continuous-other}, Asian options with arithmetic mean cannot be valued by closed-form formulae in the Black–Scholes model, and their valuation requires the use of numerical methods. Here we consider a tree method for pricing these types of options.

The main barrier to applying the Cox–Ross–Rubinstein method \cite{Cox1979} introduced in Chapter \ref{cha:models} to Asian options with arithmetic averages is the exponential increase in the number of paths that the underlying may take, and this increases the computational difficulty very quickly as we increase the number of time steps.

\paragraph{Tree methods}
Alternative feasible approaches were proposed by Hull and White (1993) \cite{Hull1993}, and Barraquand and Pudet (1996) \cite{Barraquand1996}. The main idea behind their procedures is to restrict the range of all the possible arithmetic averages to a set of representative values. These values are selected in order to span all the possible values of the averages achievable at each node of the tree. The price is then computed by a backward induction procedure in which the prices associated with averages not included in the set of representative values are obtained by interpolation. Both of these methods reduce the computational complexity to $ O(n^3) $, $ n $ being the number of steps. Nevertheless, the advantage of speed is offset by the fact that it is difficult to control the precision of the approximations and the convergence to the continuous value. This was highlighted by Forsyth \emph{et al} \cite{Forsyth2002} in 2002. Forsyth \emph{et al} also proved that a procedure of order $ O(n^{\frac{7}{2}}) $ is necessary in order to assure the convergence of these algorithms.

Later, Chalasani \emph{et al} (1999) \cite{Chalasani1999} proposed a totally different approach, which allowed them to obtain thin upper and lower bounds on the exact Cox–Ross–Rubinstein binomial price for American Asian options. Their method requires a forward procedure and a backward induction. This algorithm significantly increases the precision of the estimates but requires a very large amount of memory and has computational complexity $ O(n^4) $.

\paragraph{PDE based methods}
All of the above were tree methods. More recently, very efficient partial differential equation (PDE)-based methods have been introduced by Vecer (2001) \cite{Vecer2001} and d'Halluin et al (2005) \cite{dHalluin2005}. In Vecer's method, the price of the Asian option is characterized by a simple one-dimensional partial differential equation which could be applied to both continuous and discrete average Asian option. The compuatational complexity is $ O(n^2) $. This approach cannot be applied to American fixed-strike Asian options, which, on the other hand, can be treated using the semi-lagrangian approach of d'Halluin \emph{et al}.

Table \ref{tab:asian-literature-review} briefly reviews the discussion above.
\begin{table}[h]
	\centering
	\caption{Pre-existing methods for Asian options}
	\label{tab:asian-literature-review}
	% \rowcolors{1}{Burlywood1}{}
	\begin{tabular}{cccl}
		\toprule
		Method  &  Type  &  Complexity  &  Remarks  \\
		\midrule
		Binomial  &  Tree  &  $ 2^n $  &  simple, accurate, convergence to continuous  \\
		Hull \& White  &  Tree  &  $ O(n^3) $  &  accuracy and convergence problems  \\
		Barraquand \& Pudet  &  Tree  &  $ O(n^3) $  &  accuracy and convergence problems  \\
		Chalasani et al  &  Tree  &  $ O(n^4) $  &  thin bounds, but very large memory  \\
		Vecer  &  PDE  &  $ O(n^2) $  &  not universally applicable  \\
		d'Halluin  &  PDE  &  NA  &  more general than Vecer \\
		\bottomrule
	\end{tabular}
\end{table}


A number of these algorithms has been implemented in Premia 13. Premia is a software designed for option pricing, hedging and financial model calibration. It has been developed by the `MathFi' team in INRIA. It is provided with its C/C++ source code and an extensive scientific documentation. More information about Premia can be found at the dedicated  website\footnote{\url{https://www.rocq.inria.fr/mathfi/Premia/}}.


\section{The exact binomial algorithm}
\label{sec:asian-binom}
The evolution of the risky asset price $ (S_t)_t $ is governed by the Black-Scholes stochastic differential equation given below.
\begin{subequations}
	\begin{align*}
		\frac{\dif S_t}{S_t}  &=  (r - q) \dif t + \sigma \dif W_t  \\
		S_0  &=  s_0
	\end{align*}
\end{subequations}

The quantities used above are defined as follows.
\begin{description}
	\item[$ S_t $] the price of the underlying risky asset at time $ t $
	\item[$ W_t $] a standard Brownian motion under the risk-neutral probability measure $ P^* $
	\item[$ r > 0 $] the instantaneous interest rate
	\item[$ q \ge 0 $] the continuous divident yield
	\item[$ \sigma > 0 $] the volatility of the risky asset
\end{description}
(Refer also to Equation \ref{eq:continuous-risky-sde-risk-neutral} and Remark \ref{rem:continuous-dividend} in Section \ref{sec:continuous} of Chapter \ref{cha:models}.)

Solving the equation, at any time $ t $, the price of the underlying risky asset is given by
\begin{equation*}
\label{eq:clq-risky-prc}
S_t = s_0 e^{ ( (r - q) - \frac{\sigma^2}{2} )t + \sigma B_t }.
\end{equation*}


\paragraph{Asian options}
Asian options are dependent on the averaging of the underlying
The price of an American Asian option of initial time 0 and maturity T is
\begin{equation}
	P(0, S_0, A_0) = \sup_{\tau \in \mathcal{T}_{[0,T]}}  \E^* \left(  e^{-r \tau} \  h(S_{\tau}, A_{\tau})  \mid  S_0 = s_0, A_0 = s_0  \right) 
\end{equation}
The quantities used in the formula are explained below.
\begin{description}
	\item[$ \mathcal{T}_{[0,T]} $] the set of all stopping times with values in $ [0, T ] $
	\item[$ h $] the payoff function
	\item[$ A_\tau $] the average of the price of the underlying asset over the period $ [0, \tau] $
\end{description}

Let $ K $ denote the strike price. The price function may be one of the following
\begin{description}
	\item[fixed Asian call] $ h = (A_T - K)_+ $
	\item[fixed Asian put] $ h = (K - A_T)_+ $
	\item[floating Asian call] $ h = (S_T - A_T)_+ $
	\item[floating Asian put] $ h = (A_T - S_T)_+ $
\end{description}

Now we consider the discrete model. If the number of time steps in the binomial tree is $ n $, then the corresponding time step is $ \Delta T = \frac{T}{n} $. The lognormal diffusion process $ (S_{i \Delta T})_{i \in [n]} $ is approximated by the Cox–Ross–Rubinstein binomial process (refer Equation \ref{eq:discrete-risky-prod-iid}).
\begin{equation*}
	S_i = s_0 \prod_{j=1}^{i} T_j  \qquad  \forall i \in [n] .
\end{equation*}

As usual, we represent the risk-neutral probability by $ p = \frac{R - d}{u - d} $, where $ u = d^{-1} = e^{\sigma \Delta T} $. We denote the effective rate of interest in each period as $ R \coloneqq e^{r \Delta T} $.

In the Cox–Ross–Rubinstein model, the price at time 0 of the Asian option of the American and European types with payoff function $ h $ is given by $ v(0, s_0, s_0) $, where the functions $ v(i, x, y) $ can be computed by the following backward dynamic programming equations.
\begin{subequations}
	\label{eq:asian-dp-eu}
	\begin{align}
		v(n, x, y)  &=  h(x,y)  \\
		v(i, x, y)  &=  \frac{1}{R} \left(  p v \left( i + 1, x u, \frac{(i + 1) y + x u}{i + 2} \right)  \right.  \\
		&  \qquad  \left. + (1-p) v \left( i + 1, x d, \frac{(i + 1) y + x d}{i + 2} \right)  \right)  \qquad   \forall i \in [n - 1]  \nonumber
	\end{align}
\end{subequations}

In case of Asian options of the American type, we modify the equations accordingly.
\begin{subequations}
	\label{eq:asian-dp-am}
	\begin{align}
		v(n, x, y)  &=  h(x,y)  \\
		v(i, x, y)  &=  \max \left\lbrace  h(x, y), \frac{1}{R} \left(  p v \left( i + 1, x u, \frac{(i + 1) y + x u}{i + 2} \right)  \right. \right. \\
		&  \qquad  \left. \left. + (1-p) v \left( i + 1, x d, \frac{(i + 1) y + x d}{i + 2} \right)  \right)  \right\rbrace  \qquad  \forall i \in [n - 1]  \nonumber
	\end{align}
\end{subequations}

The payoff is a function of the average, which is clearly path-dependent. Thus, the option is path-dependent, and the corresponding price tree is non-recombinant. This makes the classical binomial method infeasible after a small number of steps. Note that the binomial tree for the underlying is always recombinant for constant volatility.


\section{The singular points method}
\label{sec:asian-method}

The price of an Asian option at each instance is a continuous function of the underlying's average. Since the number of paths to a node in a binomial tree is finite, we have that at each node of the underlying's binomial tree, the option price may be represented as a piecewise-linear, continuous, convex function of the average. We shall develop the theoretical idea in this section. In the subsequent section, we shall see that the nature of the function allows us to make approximations with \emph{a priori} error bounds.


\begin{dfn}[singular points and singular values] \label{def:asian-sp}
	Let $ P = (P_i)_{i \in [n]} = ( (x_i, y_i) )_{i \in [n]} $, $ n \in \mathbb{N} $ be a sequence of points such that
	\begin{subequations} \label{eq:asian-conditions}
		\begin{align}
			a =& x_0 < x_1 < \dots < x_{n-1} < x_n = b \\
			\label{eq:asian-condition-slope}
			m_{i+1} :=& \frac{y_{i+1} - y_{i}}{x_{i+1} - x_{i}} \le \frac{y_{i+2} - y_{i+1}}{x_{i+2} - x_{i+1}} = m_{i+2} \qquad \forall i \in \{ 1, \dots, n-1 \}
		\end{align}
	\end{subequations}
	
	Let $ f:[a,b] \to [0, \infty) $ be the function obtained by linear interpolation of the points in $P$. From the definition of $f$ and	 \ref{eq:asian-condition-slope}, the function is continuous, piecewise-linear and convex.
	
	Then, the elements of $P$ are called \emph{singular points of $f$} and the abscissae $ \{ x_i \}_{i \in [n]} $ are called \emph{singular values of $f$}.
\end{dfn}


\begin{rem}
	\label{rem:asian-char}
	We note that the singular points characterise such a function completely. This can be seen from the following representation of the function.
	\begin{equation}
		\label{eq:asian-function-repr}
		f(x) = y_0 + \sum_{i=1}^n [ m_i ( \min \{x_{i}, x \} - \min \{ x_{i-1}, x \} ) ]
	\end{equation}
	Where $ m_{i+1} = \frac{y_{i+1} - y_{i}}{x_{i+1} - x_{i}} $ represents the slope of the function between $ x_{i} $ and $ x_{i+1} $.
\end{rem}

\begin{rem}
	From the conditions \ref{eq:asian-conditions}, we get
	\begin{equation*}
		y_0 < y_1 < \dots < y_{n-1} < y_n
	\end{equation*}
	So it is equivalent to sort points using either abscissae or ordinates.
\end{rem}



\subsection{Upper estimates}
\label{subsec:asian-upper-estimates}

The following lemmas shall provide us with the necessary framework for upper and lower estimates for approximations on the functions generated by singular points.

\begin{lmm}[Upper estimate]
	\label{lmm:asian-upper-estimate}
	Let $ f:[a,b] \to [0, \infty) $ be a continuous, piecewise-linear, convex function characterised by the singular points $ P = ( (x_i, y_i) )_{i \in [n]} $. Then, if a point $ (x_j, y_j), j \in \{ 1, \dots, n-1\} $ is removed from the sequence, the function $ f_u: [a,b] \to [0, \infty) $ obtained by the new sequence $ (P_i)_{i \in [n] \setminus \{ j \}} $ is also continuous, piecewise-linear and convex, and
	\begin{equation}
		f_u(x) \ge f(x) \qquad \forall x \in [a,b]
	\end{equation}
\end{lmm}

\begin{proof}
	By construction, $ \forall x \notin ( x_{j-1} , x_{j+1} ) $, we have $ f_u(x) = f(x) $.
	
	Again, by construction, $ \forall x \in ( x_{j-1} , x_{j+1} ), f_u(x) = (1-t) f(x_{j-1}) + t f(x_{j+1}) $, where $ t = \frac{ x - x_{j-1} }{ x_{j+1} - x_{j-1} } $.
	
	Now, we have:
	\begin{alignat*}{9}
		          && x_{j-1}  & <  \qquad x          && <  x_{j+1} \\
		\implies  &&       0  & <  \quad x - x_{j-1} && <  x_{j+1} - x_{j-1} \\
		\implies  &&       0  & <  \frac{ x - x_{j-1} }{ x_{j+1} - x_{j-1} } && <  1 \\
		\implies  &&       0  & <  \qquad t          && <  1
	\end{alignat*}
	
	$f$ is convex $\implies \forall t \in (0,1), \; f( (1-t) x_{j-1} + t x_{j+1} ) < (1-t) f(x_{j-1}) + t f(x_{j+1}) $.
	
	Thus, $ f_u(x) \ge f(x) \; \forall x \in [a,b]$.
\end{proof}

\begin{figure}[h]
	\centering
	
	\definecolor{ffxfqq}{rgb}{1.,0.4980392156862745,0.}
	\definecolor{ffqqqq}{rgb}{1.,0.,0.}
	\definecolor{cqcqcq}{rgb}{0.7529411764705882,0.7529411764705882,0.7529411764705882}
	\definecolor{yqyqyq}{rgb}{0.5019607843137255,0.5019607843137255,0.5019607843137255}
	\definecolor{zzttqq}{rgb}{0.6,0.2,0.}
	\definecolor{eqeqeq}{rgb}{0.8784313725490196,0.8784313725490196,0.8784313725490196}
	
	\begin{tikzpicture}[line cap=round,line join=round,>=triangle 45,x=0.8cm,y=0.8cm]
		\draw [color=eqeqeq,dotted, xstep=1.6cm,ystep=1.6cm] (0.,0.) grid (17.,13.);
		\draw[->,color=black] (0.,0.) -- (17.,0.);
		\foreach \x in {,2.,4.,6.,8.,10.,12.,14.,16.}
		\draw[shift={(\x,0)},color=black] (0pt,2pt) -- (0pt,-2pt);
		\draw[color=black] (16.66408269434807,0.08316246397131088) node [anchor=south west] { x};
		\draw[->,color=black] (0.,0.) -- (0.,13.);
		\foreach \y in {,2.,4.,6.,8.,10.,12.}
		\draw[shift={(0,\y)},color=black] (2pt,0pt) -- (-2pt,0pt);
		\draw[color=black] (0.10395306283371611,12.541470198511039) node [anchor=west] { y};
		\clip(0.,0.) rectangle (17.,13.);
		\draw [line width=1.2pt,color=yqyqyq] (8.,5.)-- (12.,8.);
		\draw [line width=1.2pt,color=yqyqyq] (16.,12.)-- (12.,8.);
		\draw [line width=1.2pt,color=yqyqyq] (8.,5.)-- (4.,3.);
		\draw [line width=1.2pt,color=yqyqyq] (4.,3.)-- (1.,2.);
		\draw [line width=0.4pt,color=cqcqcq] (12.,8.)-- (12.,0.);
		\draw [line width=0.4pt,color=cqcqcq] (8.,5.)-- (8.,0.);
		\draw [line width=0.4pt,color=cqcqcq] (4.,3.)-- (4.,0.);
		\draw [line width=0.4pt,color=cqcqcq] (1.,2.)-- (1.,0.);
		\draw [dash pattern=on 2pt off 2pt,color=ffqqqq] (4.,3.)-- (12.,8.);
		\draw [color=ffxfqq] (8.,5.)-- (8.,5.5);
		\draw [line width=0.4pt,color=cqcqcq] (16.,12.)-- (16.,0.);
		\begin{scriptsize}
			\draw [fill=zzttqq] (16.,12.) circle (1.5pt);
			\draw[color=zzttqq] (16.227479830446466,11.294033238941376) node {$P_5$};
			\draw [fill=zzttqq] (12.,8.) circle (1.5pt);
			\draw[color=zzttqq] (12.19410099249828,7.239863120339971) node {$P_4$};
			\draw [fill=zzttqq] (8.,5.) circle (1.5pt);
			\draw[color=zzttqq] (8.223093992250325,4.474711193293883) node {$P_3$};
			\draw [fill=zzttqq] (4.,3.) circle (1.5pt);
			\draw[color=zzttqq] (4.2105057668688834,2.4788120579824215) node {$P_2$};
			\draw [fill=zzttqq] (1.,2.) circle (1.5pt);
			\draw[color=zzttqq] (1.21665755725786,1.647187418269313) node {$P_1$};
			\draw [fill=zzttqq] (16.,0.) circle (1.5pt);
			\draw[color=zzttqq] (16.18589860531298,0.5452847706494437) node {$x_5$};
			\draw [fill=zzttqq] (12.,0.) circle (1.5pt);
			\draw[color=zzttqq] (12.152519767364792,0.524494154656616) node {$x_4$};
			\draw [fill=zzttqq] (8.,0.) circle (1.5pt);
			\draw[color=zzttqq] (8.160722154550095,0.5037035386637883) node {$x_3$};
			\draw [fill=zzttqq] (4.,0.) circle (1.5pt);
			\draw[color=zzttqq] (4.2105057668688834,0.5037035386637883) node {$x_2$};
			\draw [fill=zzttqq] (1.,0.) circle (1.5pt);
			\draw[color=zzttqq] (1.1958669446911165,0.5660753866422714) node {$x_1$};
			\draw [fill=ffqqqq] (8.,5.5) circle (1.5pt);
			\draw[color=ffqqqq] (7.99439725401615,6.075588624741618) node {$Q_3$};
		\end{scriptsize}
	\end{tikzpicture}
	
	\caption{Upper estimate \\ Illustration of Lemma \ref{lmm:asian-upper-estimate} with $ j = 3 $}
	\label{fig:upper-estimate}
\end{figure}



\subsection{Lower estimates}
\label{subsec:asian-lower-estimates}

\begin{lmm}[Lower estimate]
	\label{lmm:asian-lower-estimate}
	Let $ f:[a,b] \to [0, \infty) $ be a continuous, piecewise-linear, convex function characterised by the singular points $ P = ( (x_i, y_i) )_{i \in [n]} $. Let $ l_{j} $ be the line segment joining points $ P_{j-1} $ and $ P_{j} $. Similarly, let $ l_{j+2} $ be the line segment joining points $ P_{j+1} $ and $ P_{j+2} $. Denote the intersection of the line segments $ l_{j} $ and $ l_{j+2} $ by $ \bar{P} = ( \bar{x}, \bar{y} ) $.
	
	Then the function $ f_d: [a,b] \to [0, \infty) $ characterised by $ (P_0, \dots, P_{j-1}, \bar{P}, P_{j+2}, \dots, P_n) $ is also continuous, piecewise-linear and convex, and
	\begin{equation}
		f_d(x) \le f(x) \qquad \forall x \in [a,b]
	\end{equation}
\end{lmm}

\begin{proof}
	First we show the convexity of $f_d$. We know that $f$ satisfies the property of increasing slopes, that is $ m_{i} \le m_{i+1} \le m_{i+2} $. Since $f_d$ is obtained from $f$ by removing the line segment $l_{j+1}$, for $f_d$ we have that $ m_{i} \le m_{i+2} $, which implies that the function $f_d$ is still convex.
	
	Secondly, to prove the inequality, we may look at the convex function $f$ as if it has been obtained by removing point $ \bar{P} $ from the convex function $f_d$. Then, if $ \bar{x} \in ( x_{j} , x_{j+1} ) $, we have, using Lemma \ref{lmm:asian-upper-estimate}, that $ f_d(x) \le f(x) \qquad \forall x \in [a,b] $.
\end{proof}

\begin{figure}[h]
	\centering
	
	\definecolor{ffxfqq}{rgb}{1.,0.4980392156862745,0.}
	\definecolor{ffqqff}{rgb}{1.,0.,1.}
	\definecolor{cqcqcq}{rgb}{0.7529411764705882,0.7529411764705882,0.7529411764705882}
	\definecolor{yqyqyq}{rgb}{0.5019607843137255,0.5019607843137255,0.5019607843137255}
	\definecolor{zzttqq}{rgb}{0.6,0.2,0.}
	\definecolor{eqeqeq}{rgb}{0.8784313725490196,0.8784313725490196,0.8784313725490196}
	\begin{tikzpicture}[line cap=round,line join=round,>=triangle 45,x=0.8cm,y=0.8cm]
		\draw [color=eqeqeq,dotted, xstep=1.6cm,ystep=1.6cm] (0.,0.) grid (17.,13.);
		\draw[->,color=black] (0.,0.) -- (17.,0.);
		\foreach \x in {,2.,4.,6.,8.,10.,12.,14.,16.}
		\draw[shift={(\x,0)},color=black] (0pt,2pt) -- (0pt,-2pt);
		\draw[color=black] (16.672398939374787,0.08316246397131087) node [anchor=south west] { x};
		\draw[->,color=black] (0.,0.) -- (0.,13.);
		\foreach \y in {,2.,4.,6.,8.,10.,12.}
		\draw[shift={(0,\y)},color=black] (2pt,0pt) -- (-2pt,0pt);
		\draw[color=black] (0.10395306283371619,12.533153952113903) node [anchor=west] { y};
		\clip(0.,0.) rectangle (17.,13.);
		\draw [line width=1.2pt,color=yqyqyq] (8.,5.)-- (12.,8.);
		\draw [line width=1.2pt,color=yqyqyq] (16.,12.)-- (12.,8.);
		\draw [line width=1.2pt,color=yqyqyq] (8.,5.)-- (4.,3.);
		\draw [line width=1.2pt,color=yqyqyq] (4.,3.)-- (1.,2.);
		\draw [line width=0.4pt,color=cqcqcq] (12.,8.)-- (12.,0.);
		\draw [line width=0.4pt,color=cqcqcq] (8.,5.)-- (8.,0.);
		\draw [line width=0.4pt,color=cqcqcq] (4.,3.)-- (4.,0.);
		\draw [line width=0.4pt,color=cqcqcq] (1.,2.)-- (1.,0.);
		\draw [dash pattern=on 2pt off 2pt,color=ffqqff] (8.,5.)-- (10.,6.);
		\draw [dash pattern=on 2pt off 2pt,color=ffqqff] (12.,8.)-- (10.,6.);
		\draw [line width=0.4pt,color=cqcqcq] (10.,6.)-- (10.,0.);
		\draw [color=ffxfqq] (10.,6.)-- (10.,6.5);
		\draw [line width=0.4pt,color=cqcqcq] (16.,12.)-- (16.,0.);
		\begin{scriptsize}
			\draw [fill=zzttqq] (16.,12.) circle (1.5pt);
			\draw[color=zzttqq] (16.23579607547318,11.28571699254424) node {$P_5$};
			\draw [fill=zzttqq] (12.,8.) circle (1.5pt);
			\draw[color=zzttqq] (12.181626624958245,7.231546873942835) node {$P_4$};
			\draw [fill=zzttqq] (8.,5.) circle (1.5pt);
			\draw[color=zzttqq] (8.231410237277029,4.487185562889577) node {$P_3$};
			\draw [fill=zzttqq] (4.,3.) circle (1.5pt);
			\draw[color=zzttqq] (4.1980313993288405,2.4704958115852884) node {$P_2$};
			\draw [fill=zzttqq] (1.,2.) circle (1.5pt);
			\draw[color=zzttqq] (1.2041831897178144,1.6388711718721796) node {$P_1$};
			\draw [fill=zzttqq] (16.,0.) circle (1.5pt);
			\draw[color=zzttqq] (16.19421485033969,0.5369685242523106) node {$x_5$};
			\draw [fill=zzttqq] (12.,0.) circle (1.5pt);
			\draw[color=zzttqq] (12.140045399824759,0.5161779082594828) node {$x_4$};
			\draw [fill=zzttqq] (8.,0.) circle (1.5pt);
			\draw[color=zzttqq] (8.169038399576799,0.4953872922666551) node {$x_3$};
			\draw [fill=zzttqq] (4.,0.) circle (1.5pt);
			\draw[color=zzttqq] (4.1980313993288405,0.4953872922666551) node {$x_2$};
			\draw [fill=zzttqq] (1.,0.) circle (1.5pt);
			\draw[color=zzttqq] (1.1833925771510712,0.5577591402451383) node {$x_1$};
			\draw [fill=zzttqq] (10.,0.) circle (1.5pt);
			\draw[color=zzttqq] (10.22730904368438,0.5161779082594828) node {$x_{34}$};
			\draw [fill=ffqqff] (10.,6.) circle (1.5pt);
			\draw[color=ffqqff] (10.206518431117638,5.443553898559651) node {$Q_{34}$};
			\draw [fill=zzttqq] (10.,6.5) circle (1.5pt);
			\draw[color=zzttqq] (10.040193530583691,7.252337489935663) node {$P_{34}$};
			
		\end{scriptsize}
	\end{tikzpicture}
	
	\caption{Lower estimate \\ Illustration of Lemma \ref{lmm:asian-lower-estimate} with $ j = 3, x_{34} = \bar{x}, P_{34} = \bar{P} $}
	\label{fig:lower-estimate}
\end{figure}


The lemmas \ref{lmm:asian-upper-estimate} and \ref{lmm:asian-lower-estimate}, will be used later to reduce both the computational complexity and the memory requirement of the algorithm by removing points or edges, effectively simplifying the function.



\subsection{Notations and conventions}
\label{subsec:asian-notations}

In this and subsequent sections, we shall use the convention that $ [n] = \{ 0, 1, 2, \dots, n \} $.

Let the number of time steps be $n$. Let $i$ denote the highlighted time step, and $j$ represent the number of up movements. In this way, we may represent any node by $ N_{i,j} $. For example, in Figure \ref{fig:asian-paths}, the node denoted by $ S_0 u^2 d $ would be represented as $ N_{3,2} $.

The price of the underlying at each node $ N_{i,j} $ is denoted by $ S_{i,j} $. Since there are $j$ up movements, there must be $ i-j $ down movements, and thus
\begin{equation} \label{eq:asian-am-ij}
	S_{i,j} = S_0 u^{j} d^{i-j} = S_0 u^{j} u^{-(i-j)} = S_0 u^{-i+2j} \qquad \forall i \in [n], \ \forall j \in [i]
\end{equation}


\begin{prp}
	The number of paths to a node $ N_{i,j} $ is $ \binom{i}{j} $.
\end{prp}

\begin{proof}
	At each point in a path, we may choose either an up movement or a down movement. To reach node $ N_{i,j} $, we much choose $j$ up movements among $i$ possibilities. The result follows immediately.
\end{proof}


We denote the number of singular points in a node $ N_{i,j} $ by $ L_{i,j} $, where $ L_{i,j} \in \left[ \binom{i}{j} \right] $. The $ l^\mathrm{th} $ average (in ascending order) ($ l \in \{ 1, \dots, L_{i,j} \} $) is denoted by $ A_{i,j}^l $, and the corresponding price by $ P_{i,j}^l $. Thus the singular points characterising the price function are $ ( ( A_{i,j}^l, P_{i,j}^l ) )_{l \in \{ 1, \dots, L_{i,j} \} } $. Note that it is possible that some paths among all the possible paths may give zero as the price for the option. 


\begin{dfn}[singular average and singular price]
	In the particular case of Asian options with arithmetic mean, each $ A_{i,j}^l $ is called a \emph{singular average} and each $ P_{i,j}^l $ is called a  \emph{singular price}.
\end{dfn}


We recall some basic definitions and derive simple results for the maximum and minimum attainable value of the averages on each node.

Let the spot rate of interest be $ r $ (constant) and the compounding be continuous. Then, the effective compounding rate in each time period $ \Delta t $ is given by $ R $ as
\begin{equation}
	\label{eq:R}
	R = e^{r \Delta t}
\end{equation}
We note that the $R$ is not an instantaneous quantity, but one which is constant on an interval of time.


\begin{dfn}[Arithmetic mean]
	The arithmetic mean of a set of numbers $ \{ S_i \}_{i \in [n]} $ is given by:
	\begin{equation}
		\label{eq:am}
		A_{n} = \frac{\sum_{i=0}^n S_i}{n+1}
	\end{equation}
\end{dfn}

In the rest of the chapter, we will assume that the average means arithmetic mean, unless otherwise stated.


\begin{dfn}[Path]
	A path is a sequence $(j_i)_{i \in [n]}$ such that $j_{i+1} \in \{ j_i,j_i+1 \}$.
\end{dfn}

\begin{eg}
	In Figure \ref{fig:asian-paths}, two paths are shown using red/thicker and blue/thick arrows. The other arrows are in grey/thin. The two paths have the same value at maturity, but give different averages.
\end{eg}


\begin{figure}[h]
	% Recombining 4-step binomial tree for Cox-Ross-Rubinstein model
	\begin{tikzpicture}
		\matrix[column sep=10mm,row sep=1mm] (tree){
			& & & & \node[term] (u4) {$S_0u^4$}; \\
			& & & \node[nterm] (u3) {$S_0u^3$}; & \\
			& & \node[nterm] (u2) {$\bm{S_0 u^2}$}; & & \node[term] (u3d) {$S_0u^3d$}; \\
			& \node[nterm] (u) {$ \bm{S_0 u} $}; & & \node[nterm] (u2d) {$\bm{S_0 u^2 d}$};\\
			\node[term] (s) {$ \bm{S_0} $}; & & \node[nterm] (ud) {$ \bm{S_0 u d} $}; & & \node[term] (u2d2) {$ \bm{S_0 u^2 d^2} $ }; \\
			& \node[nterm] (d) {$ \bm{S_0 d} $}; & &	\node[nterm] (ud2) {$\bm{S_0 u d^2}$};\\
			& & \node[nterm] (d2) {$S_0d^2$}; & & \node[term] (ud3) {$S_0ud^3$}; \\
			& & & \node[nterm] (d3) {$S_0d^3$}; & \\
			& & & & \node[term] (d4) {$S_0d^4$}; \\
		};
		% Lines out of s
		\draw[->,red,ultra thick] (s) -- (u) node[midway,above,sloped] {$p_u$};
		\draw[->,blue,thick] (s) -- (d) node[midway,below,sloped] {$p_d$};
		% Lines out of u
		\draw[->,red,ultra thick] (u) -- (u2) node[midway,above,sloped] {$p_u$};
		\draw[->,gray] (u) -- (ud) node[midway,above,sloped] {$p_d$};
		% Lines out of d
		\draw[->,blue,thick] (d) -- (ud) node[midway,below,sloped] {$p_u$};
		\draw[->,gray] (d) -- (d2) node[midway,below,sloped] {$p_d$};
		% Lines out of u2
		\draw[->,gray] (u2) -- (u3) node[midway,above,sloped] {$p_u$};
		\draw[->,red,ultra thick] (u2) -- (u2d) node[midway,above,sloped] {$p_d$};
		% Lines out of ud
		\draw[->,gray] (ud) -- (u2d) node[midway,above,sloped] {$p_u$};
		\draw[->,blue,thick] (ud) -- (ud2) node[midway,below,sloped] {$p_d$};
		% Lines out of d2
		\draw[->,gray] (d2) -- (ud2) node[midway,below,sloped] {$p_u$};
		\draw[->,gray] (d2) -- (d3) node[midway,below,sloped] {$p_d$};
		% Lines out of u3
		\draw[->,gray] (u3) -- (u4) node[midway,above,sloped] {$p_u$};
		\draw[->,gray] (u3) -- (u3d) node[midway,above,sloped] {$p_d$};
		% Lines out of u2d
		\draw[->,gray] (u2d) -- (u3d) node[midway,above,sloped] {$p_u$};
		\draw[->,red,ultra thick] (u2d) -- (u2d2) node[midway,above,sloped] {$p_d$};
		% Lines out of ud2
		\draw[->,blue,thick] (ud2) -- (u2d2) node[midway,below,sloped] {$p_u$};
		\draw[->,gray] (ud2) -- (ud3) node[midway,below,sloped] {$p_d$};
		% Lines out of d3
		\draw[->,gray] (d3) -- (ud3) node[midway,below,sloped] {$p_u$};
		\draw[->,gray] (d3) -- (d4) node[midway,below,sloped] {$p_d$};
	\end{tikzpicture}
	\caption{A 4-step lattice with different paths to the same point}
	\label{fig:asian-paths}
\end{figure}


\begin{thm}[Path inequality]
	\label{thm:asian-up-dn-path}
	Let there be two paths $\alpha$ and $\beta$, such that $S_{i,j_i^\alpha} \ge S_{i,j_i^\beta} \; \forall i$, where $ ( j_i^\alpha )_{i \in [n]} $ and $ ( j_i^\beta )_{i \in [n]} $ denote the paths as defined above. Denote the corresponding averages by $A^\alpha$ and $A^\beta$, respectively. Then $ A^\alpha \ge A^\beta $.
\end{thm}

\begin{proof}
	Clearly if $S_{i,j_i^\alpha} = S_{i,j_i^\beta} \; \forall i$, then $A^\alpha = A^\beta$.
	
	We only need to show the result in the case of strict inequality at one time.
	Let $ S_{i,j_i^\alpha} = S_{i,j_i^\beta} \; \forall i \in [n] \setminus \{l\} $, and $ S_{l,j_l^\alpha} > S_{l,j_l^\beta}$.
	
	Now, from equation \ref{eq:am}, we have:
	\begin{align*}
		(n+1) A_{n,j}^\alpha &= \sum_{i=0}^{l-1} S_{i,j_i} + S_{l,j_l^\alpha} + \sum_{i=l+1}^{n} S_{i,j_i} \\
		(n+1) A_{n,j}^\beta &= \sum_{i=0}^{l-1} S_{i,j_i} + S_{l,j_l^\beta} + \sum_{i=l+1}^{n} S_{i,j_i} \\
		\implies (n+1) \left(A_{n,j}^\alpha - A_{n,j}^\beta\right) &= S_{l,j_l^\alpha} - S_{l,j_l^\beta} \\
												 &= S_{l-1,j_{l-1}} u_l - S_{l-1,j_{l-1}} d_l \\
												 &= S_{l-1,j_{l-1}} (u_l - d_l) > 0 \qquad (u_l > d_l \text{ by definition}) \\
		\implies A_{n,j}^\alpha > A_{n,j}^\beta
	\end{align*}
	
	Iterating this procedure, we obtain the general case.
\end{proof}


\begin{rem}
	The path $\alpha$ signifies a path \emph{above} and $\beta$ a path \emph{below} in the usual depiction of the binomial tree (the up movement shown above the down movement). Thus, a path which never goes below another cannot have a lower arithmetic mean than the other.
\end{rem}


\begin{crr}
	\label{crr:asian-up-dn-path}
	At each node $ N(i,j) $, the following hold:
	\begin{enumerate}
	\item The minimum average possible $ A_{i,j}^{\min} $ is attained by the path corresponding to the path corresponding to the path with $(i-j)$ down movements followed by $j$ up movements, and
		\begin{equation}	\label{eq:asian-Amin}
			A_{i,j}^{\min} = \frac{S_0}{i+1} \left( \frac{1 - d^{i-j+1}}{1-d} + d^{i-j} u \frac{1 - u^{j}}{1-u} \right)
		\end{equation}
	\item The maximum average possible $ A_{i,j}^{\max} $ is attained by the path corresponding to the path with $j$ up movements followed by $(i-j)$ down movements, and
		\begin{equation} \label{eq:asian-Amax}
			A_{i,j}^{\max} = \frac{S_0}{i+1} \left( \frac{1 - u^{j+1}}{1-u} + u^{j} d \frac{1 - d^{i-j-1}}{1-d} \right)
		\end{equation}
	\end{enumerate}
\end{crr}

\begin{proof}
	We show the proof only for the case of the maximum, since the case of the minimum can be shown by using the exact same argument.
	
	From Theorem \ref{thm:asian-up-dn-path}, the result about path with the maximum average holds directly, since there cannot be a path above the one given by $j$ up movements followed by $(i-j)$ down movements.
	
	The subsequent formula may be derived as follows.
	\begin{align*}
		(i+1) A_{i,j}^{\max} &= \underbrace{ ( S_0 + S_0 u + S_0 u^2 + \dots + S_0 u^j ) }_\text{up movement} + \underbrace{ ( S_0 u^j d + S_0 u^j d^2 + \dots + S_0 u^j d^{i-j} ) }_\text{down movement} \\
		&= S_0 ( (1 + u + u^2 + \dots + u^j ) + u^j d ( 1 + d + \dots + d^{i-j-1} ) ) \\
		&= S_0 \left( \sum_{k=0}^j u^k + u^j d \sum_{k=0}^{i-j-1} d^k \right) \\
		&= S_0 \left( \frac{1 - u^{j+1}}{1-u} + u^{j} d \frac{1 - d^{i-j-1}}{1-d} \right) \qquad \text{(Geometric series)} \\
		\implies A_{i,j}^{\max} &= \frac{S_0}{i+1} \left( \frac{1 - u^{j+1}}{1-u} + u^{j} d \frac{1 - d^{i-j-1}}{1-d} \right)
	\end{align*}
\end{proof}

Table \ref{tab:asian-notations} summarises the discussion above.

\begin{table}[h]
	\centering
	\caption{Summary of notations}
	\label{tab:asian-notations}
%	\rowcolors{1}{Burlywood1}{}
	\begin{tabular}{cccl}
		\toprule
		Symbol & Range & Formula & Description \\
		\midrule
		$ i $ & $ [ n ] $ & & highlighted time step \\
		$ j $ & $ [ i ] $ & & number of up movements \\
		$ N_{i,j} $ & & & node fixed by $ (i,j) $ \\
		$ S_{i,j} $ & $ [0, \infty) $ & Eq \ref{eq:asian-am-ij} & value of the underlying at node $ N_{i,j} $ \\
		$ L_{i,j} $ & $ \left[ \binom{i}{j} \right] $ & & number of singular points in node $ N_{i,j} $ \\
		$ l $ & $ \{ 1, \dots, L \} $ & & index for points in ascending order of averages \\
		$ A_{i,j}^{\min} $ & $ [0, \infty) $ & Eq \ref{eq:asian-Amin} & minimum average attainable for node $ N_{i,j} $ \\
		$ A_{i,j}^{\max} $ & $ [0, \infty) $ & Eq \ref{eq:asian-Amax} & maximum average attainable for node $ N_{i,j} $ \\
		$ A_{i,j}^l $ & $ \left[ A_{i,j}^{\min}, A_{i,j}^{\max} \right] $ & Eq \ref{eq:am} & $ l^\mathrm{th} $ singular average of node $ N_{i,j} $ \\
		$ P_{i,j}^l $ & & & price corresponding to the average $ A_{i,j}^l $ \\
		$ (A_{i,j}^l, P_{i,j}^l) $ & & & $ l^\mathrm{th} $ singular point of node $ N_{i,j} $ \\
		\bottomrule
	\end{tabular}
\end{table}



\section{Fixed-strike Asian options of the European type}
\label{sec:fixed-strike-eu}

This is the simplest case. For this type of option, the pay-off at maturity is dependent only on (some type of) average $ A_T $ at maturity $ T $ and a fixed constant $ K $, and is given by the function
\begin{equation}
	\label{eq:asian-price-eu-asian-am}
	P_T = (A_T - K)_+
\end{equation}

In each node of the binomial tree, we have a set of possible averages depending on the paths which may be taken to arrive at the node, and prices corresponding to each of those averages. We shall show that these points satisfy condition \ref{eq:asian-conditions}, so they completely characterise the price function. So we consider not only averages that are effectively achievable but all the possible averages between the minimum and maximum realized at that point. This gives us the continuous representation of prices. The intuitive idea is that as the time step is reduced to zero, this function converges to the price function of the continuous time model.

We start with the prices at maturity, and proceed using backward iteration. The exact details are explained below.


\paragraph{At maturity ($ i = n $)}

From equations \ref{eq:asian-Amin} and \ref{eq:asian-Amax}, putting $i = n$, we get
\begin{align*}
	A_{n,j}^{\min} &= \frac{S_0}{n+1} \left( \frac{1 - d^{n-j+1}}{1-d} + d^{n-j} u \frac{1 - u^{j}}{1-u} \right) \\
	A_{n,j}^{\max} &= \frac{S_0}{n+1} \left( \frac{1 - u^{j+1}}{1-u} + u^{j} d \frac{1 - d^{n-j-1}}{1-d} \right)
\end{align*}

In defining the price function, we note that three cases may arise.
\begin{itemize}
\item $ j \in \{ 0, n \} $ \\
	In this case, there can be only one path to these nodes, so there is only one average, implying one price and one singular point.
	
\item $ j \notin \{ 0, n \} $ and $ K \in ( A_{n,j}^{\min}, A_{n,j}^{\max} ) $ \\	
	In this case, the price function is characterised by three singular points ($ L_{i,j} = 3 $), $ ( A_{n,j}^l , P_{n,j}^l )_{l \in \{ 1, 2, 3 \} } $, since we need to compare the averages with the fixed strike price $ K $. The points are as follows.  \\
	\begin{equation}
		\label{eq:asian-price-maturity-kin}
		\begin{aligned}
			( A_{n,j}^1 , P_{n,j}^1 ) &= ( A_{n,j}^{\min} , 0 ) \\
			( A_{n,j}^2 , P_{n,j}^2 ) &= ( K , 0 ) \\
			( A_{n,j}^3 , P_{n,j}^3 ) &= ( A_{n,j}^{\max} , A_{n,j}^{\max} - K ) \\
		\end{aligned}
	\end{equation} \label{eq:asian-price-maturity-kout}
	
\item $ j \notin \{ 0, n \} $ and $ K \notin ( A_{n,j}^{\min}, A_{n,j}^{\max} ) $ \\
	In this case, the price function is characterised by only two singular points ($ L_{i,j} = 2 $), $ ( A_{n,j}^l , P_{n,j}^l )_{l \in \{ 1, 2 \} } $, which are as follows. \\
	\begin{equation}
		\label{eq:asian-price-maturity-knotin}
		\begin{aligned}
			( A_{n,j}^1 , P_{n,j}^1 ) &= ( A_{n,j}^{\min} , ( A_{n,j}^{\min} - K )_+ ) \\
			( A_{n,j}^2 , P_{n,j}^2 ) &= ( A_{n,j}^{\max} , ( A_{n,j}^{\max} - K )_+ ) \\
		\end{aligned}
	\end{equation}
\end{itemize}

\begin{lmm}[Price function at maturity ($ i < n $)]
	\label{lmm:asian-pr-maturity}
	At each node at maturity, the price function $ { v_{n,j}: \left[ A_{n,j}^{\min}, A_{n,j}^{\max} \right] \to \left[ ( A_{n,j}^{\min} - K )_+ , ( A_{n,j}^{\max} - K )_+ \right] } $ defined as $ v_{n,j}(A) = (A - K)_+ $ is continuous, piecewise-linear and convex.
\end{lmm}
\begin{proof}
	The singular points satisfy the conditions \ref{eq:asian-conditions}. So for each $ A \in \left[ A_{n,j}^{\min}, A_{n,j}^{\max} \right] $, the price function ${ v_{n,j}(A) }$ characterised by the singular points is continuous, piecewise-linear and convex by remark \ref{rem:asian-char}.
\end{proof}



\paragraph{Before maturity ($ i < n $)}

\begin{lmm}[Price function at any node]
	\label{lmm:asian-dsc-expt}
	At any node $ N_{i,j} $, the price function $ v_{i,j}: \left[ A_{i,j}^{\min}, A_{i,j}^{\max} \right] \to [0, \infty) $ is continuous, piecewise-linear and convex.
\end{lmm}

\begin{proof}
	We shall prove this using backward induction, the base case at maturity being true by virtue of Lemma \ref{lmm:asian-pr-maturity}.
	We now consider step $ i = n-1 $. Let $A_u$ and $A_d$ respectively represent the averages after an up and down movement corresponding to an average $A$. From equation \ref{eq:am}, we get
	\begin{subequations}
		\label{eq:asian-av-up-dn}
		\begin{align}
			A_u &= \frac{ (i+1) A + S_0 u^{-i+2j+1} }{ i+1 } \\
			A_d &= \frac{ (i+1) A + S_0 u^{-i+2j-1} }{ i+1 }
		\end{align}
	\end{subequations}
	Since the options is of the European type, applying the no-arbitrage condition, the price function $ v_{i,j}: \left[ A_{i,j}^{\min}, A_{i,j}^{\max} \right] \to [0, \infty) $ is obtained by considering the discounted expectation value.
	\begin{equation}
		\label{eq:asian-dsc-expt}
		v_{i,j}(A) = \frac{1}{R} \left[ p v_{i+1,j+1}(A_u) + (1 - p) v_{i+1,j}(A_d) \right]
	\end{equation}
	From equation \ref{eq:asian-av-up-dn}, we get that $A_u$ and $A_d$ are linear functions of $A$. Thus, $ v_{i+1,j+1}(A_u) = v_{n,j+1}(A_u)$ and $ v_{i+1,j}(A_d) = v_{n,j}(A_d) $ are piecewise-linear convex continuous functions of $A_u$ and $A_d$ respectively. Thus, $ v_{i+1,j+1} $ and $ v_{i+1,j} $ may be seen as a linear combination of the above functions, and is thus piecewise-linear, convex and continuous itself. Again, from equation \ref{eq:asian-dsc-expt}, we get that $v_{i,j}$ is a convex combination of such functions, and the proof is complete.
	
	We showed that if at time $ i+1 $, if the price function is continuous, piecewise-linear and convex, so is it for time $ i $. Since this is true for $ i+1 = n $, the same logic applied iteratively proves that the functions retain the characteristics for all $ i \in [n] $.
\end{proof}


\begin{rem}
	From Lemma \ref{lmm:asian-dsc-expt}, we see that the price function may be characterised by singular points.
\end{rem}



\subsection{Evaluation of singular points}
\label{subsec:asian-eu-eval}

The evaluation of singular points for any node $ N_{i,j} $ is done by the following algorithm, which works in a backward fashion in time, starting from the maturity.

We note that for the only influencing nodes for the node $ N_{i,j} $ are $ N_{i+1,j+1} $ and $ N_{i+1,j} $. Thus we need to calculate the price of the option for each singular average belonging to either of these nodes.


\paragraph{Up movement}

First we take each singular average $ A_{i+1,j}^l $ belonging to $ N_{i+1,j} $ and project it to $ N_{i,j} $ via the relation
\begin{equation}
	\label{eq:asian-proj-up}
	B^l = \frac{ ( i+2) A_{i+1,j}^l - S_0 u^{-i+2j-1} }{ i+1 }
\end{equation}
Thus, $ B^l $ is that average which after a down movement of the asset gives us the average $ A_{i+1,j}^l $.

Next, we note that $ B^l $ is an increasing function of $ l $, since a higher average at time step $ i $ would yield a higher average at time $ i+1 $. This in turn implies the following:
\begin{itemize}
\item $ B^{L_{i+1,j}} = A_{i+i,j}^{\max} \; \forall j $
\item $ B^1 \notin \left[ A_{i+i,j}^{\min}, A_{i+i,j}^{\max} \right] \ \forall j \in \{1, \dots, i-1 \} $
\end{itemize}
Each $ B^l \in \left[ A_{i,j}^{\min}, A_{i,j}^{\max} \right] $ becomes the singular average of $ N_{i,j} $.

In this way, we have determined the first coordinate of the singular points. We need to determine the second coordinate, or the prices $ v_{i,j}(B^l) $, $ \forall \left[ A_{i,j}^{\min}, A_{i,j}^{\max} \right] $, in order to determine the singular points completely. The idea is to calculate the discounted expected value of the price corresponding to each average $ B^l $ at $ N_{i,j} $. In order to be able to do this, we need the prices corresponding to the average projected to the node $ N_{i+1,j+1} $.

We consider an up movement of the underlying asset from node $ N_{i,j} $. In this case, $ B^l $ transforms into the average $ B^l_u = \left( (i+1) B^l + S_0 u^{-i+2j+1} \right) / ( i+2 ) $. Clearly, this average cannot belong to the set of averages associated with the node $ N_{i+1,j+1} $. Thus, we need to find the index $s$ such that $ B^l_u \in \left[ A_{i+1,j+1}^{s} , A_{i+1,j+1}^{s+1} \right] $. In the intervals the price function is linear, and thus we have
\begin{equation}
	\label{eq:asian-up-lint}
	v_{i+1,j+1} \left( B^l_u \right) = \frac{ P_{i+1,j+1}^{s+1} - P_{i+1,j+1}^{s} }{ A_{i+1,j+1}^{s+1} - A_{i+1,j+1}^{s} } \left( B^l_u - A_{i+1,j+1}^{s} \right) + P_{i+1,j+1}^{s}
\end{equation}

We follow this up by calculating the price associated with the singular value $ B^l $ by evaluating the discounted expectation value.
\begin{equation}
	\label{eq:asian-up-pr}
	v_{i,j}( B^l ) = \frac{1}{R} \left[ p v_{i+1,j+1} \left( B^l_u \right) + (1 - p) v_{i+1,j} \left( A_{i+1,j}^l \right) \right]
\end{equation}

Figure \ref{fig:asian-2tr-up} depicts the idea.
\begin{figure}[h]
	\begin{tikzpicture}
	\matrix (tree) [column sep=25mm, row sep=1mm]{
		\node[header] (t0) {$ t = i $};  &  \node[header] (t1) {$ t = i+1 $}; \\
		&  \node[term] (u) {$ B^l_u $}; \\
		\node[term] (s) {$ B^l $};  &  \\
		&  \node[term] (d) {$ A_{i+1,j}^l $}; \\
	};
	\draw[->] (s) -- (u) node[midway,above,sloped] {Step 2};
	\draw[->] (d) -- (s) node[midway,below,sloped] {Step 1};
	\end{tikzpicture}
	
	\caption{Up movement}
	\label{fig:asian-2tr-up}
\end{figure}



\paragraph{Down movement}

We now proceed to formulate the theory for the downward movement in the exact same fashion. Define the new average $ C^l $ at the node $ N_{i,j} $ via the relation
\begin{equation}
	\label{eq:asian-proj-dn}
	C^l = \frac{ ( i+2) A_{i+1,j+1}^l - S_0 u^{-i+2j+1} }{ i+1 }
\end{equation}

Again, we note that
\begin{itemize}
\item $ C^1 = A_{i,j}^{\min} \ \forall j $
\item $ C^{L_{i+1,j+1}} \notin \left[ A_{i,j}^{\min}, A_{i,j}^{\max} \right] \ \forall j \in \{1, \dots, i-1 \} $
\item $ C^l_d = \left( (i+1) C^l + S_0 u^{-i+2j-1} \right) / ( i+2 ) $
\end{itemize}
Each $ C^l \in \left[ A_{i,j}^{\min}, A_{i,j}^{\max} \right] $ becomes the singular average of $ N_{i,j} $.

For $ v_{i,j}( C^l ) $, $ \forall \left[ A_{i,j}^{\min}, A_{i,j}^{\max} \right] $, we now have the following.
\begin{equation}
	\label{eq:asian-dn-lint}
	v_{i+1,j+1} \left( C^l_d \right) = \frac{ P_{i+1,j}^{s+1} - P_{i+1,j}^{s} }{ A_{i+1,j}^{s+1} - A_{i+1,j}^{s} } \left( C^l_d - A_{i+1,j}^{s} \right) + P_{i+1,j}^{s}
\end{equation}

\begin{equation}
	\label{eq:asian-dn-pr}
	v_{i,j}( C^l ) = \frac{1}{R} \left[ p v_{i+1,j+1} \left( A_{i+1,j+1}^l \right) + (1 - p) v_{i+1,j} \left( C^l_d \right) \right]
\end{equation}

Figure \ref{fig:asian-2tr-dn} depicts the idea.
\begin{figure}[h]
	\begin{tikzpicture}
	\matrix (tree) [column sep=25mm, row sep=1mm]{
		\node[header] (t0) {$ t = i $};  &  \node[header] (t1) {$ t = i+1 $}; \\
		&  \node[term] (u) {$ A_{i+1,j+1}^l $}; \\
		\node[term] (s) {$ C^l $};  &  \\
		&  \node[term] (d) {$ C^l_d $}; \\
	};
	\draw[->] (u) -- (s) node[midway,above,sloped] {Step 1};
	\draw[->] (s) -- (d) node[midway,below,sloped] {Step 2};
	\end{tikzpicture}
	
	\caption{Down movement}
	\label{fig:asian-2tr-dn}
\end{figure}


\paragraph{Aggregation}

Now we have the singular points for both up and down movements. We sort these points in ascending order of the first coordinate, i.e. the averages $ B^l $ and $ C^l $ that belong to $ \left[ A_{i,j}^{\min}, A_{i,j}^{\max} \right] $. These is an exhaustive list of all the singular points in the node (by construction). We note that $ L_{i,j} \le L_{i+1,j} + L_{i+1,j+1} - 2 $.

This procedure is applied to all nodes, starting from maturity and proceeding backwards. At the `edge' nodes $ N_{i,0} $ and $ N_{i,i} $, there is only one singular point whose price is given as follows
\begin{subequations}
	\label{eq:asian-terminal-nodes}
	\begin{align}
		P_{i,0}^1 &= \frac{1}{R} \left[ p P_{i+1,0}^1 + (1 - p) P_{i+1,1}^1 \right] \\
		P_{i,i}^1 &= \frac{1}{R} \left[ p P_{i+1,i+1}^1 + (1 - p) P_{i+1,i}^{L_{i+1,i}} \right]
	\end{align}
\end{subequations}

Thus we have a complete description of the price function at each node of the binomial tree. The price $ P_{0,0}^1 $ is exactly the binomial price relative to the tree with $n$ steps of a fixed-strike European call option.


\section{Fixed-strike Asian options of the American type}
\label{sec:fixed-strike-am}

We now consider the American case. At maturity we have the same situation as in the European case. The price function is $ v_{n,j} (A) = (A - K)_+ $ for $ A \in [ A_{n,j}^{\min}, A_{n,j}^{\max} ] $, and it is characterized by the same singular points.

Consider the step $ i = n - 1 $. At the node $ N_{i,j} $, we first compute, by using the procedure described in the previous section (European case), the singular points associated with this node, thus obtaining the continuation value function $ v_{n,j}^c (A) $. But now, we must also account for exercise rights at this time. Thus, we have
\begin{equation}
	v_{n,j} (A)  =  \max \{ \underbrace{A - K}_{\text{exercise}}, \underbrace{v_{n,j}^c (A)}_{\text{hold}} \} .
\end{equation}

Note that $ v_{n,j} (A) $ is still a piecewise-linear convex function, since $ \max $ is a convex function. For this reason, we can again characterize it by its singular points. In order to compute the singular points associated with the American case, we first remark that the slopes characterizing the piecewise-linear convex function $ v_{n,j}^c (A) $ are all smaller than 1. This can be seen from Equations \ref{eq:asian-av-up-dn} and \ref{eq:asian-dsc-expt} and by differentiating $ v_{n,j}^c (A) $ in the open intervals $ \left(  A_{i,j}^{l}, A_{i,j}^{l+1}  \right)  \  \forall l \in \{ 1, 2, \dots, L_{i,j} \} $. Hence there are two possible cases, as follows.

\begin{enumerate}
	\item $ A_{i,j}^{\max} - K  \le  v_{i,j}^c ( A_{i,j}^{\max} ) $. This implies that $ v_{i,j} \equiv v_{i,j}^c $. The singular points remain the same.
	\item $ A_{i,j}^{\max} - K  <  v_{i,j}^c ( A_{i,j}^{\max} ) $. This case has two subcases, as follows.
	\begin{enumerate}
		\item $ A_{i,j}^{\min} - K  \ge  v_{i,j}^c ( A_{i,j}^{\min} ) $. This implies that $ v_{i,j} (A) = (A - K), \  \forall A \in [ A_{i,j}^{\min}, A_{i,j}^{\max} ] $, so the only singular points points are $ \left(  A_{i,j}^{\min}, A_{i,j}^{\min} - K  \right) $ and $ \left(  A_{i,j}^{\max}, A_{i,j}^{\max} - K  \right) $.
		\item $ A_{i,j}^{\min} - K  <  v_{i,j}^c ( A_{i,j}^{\min} ) $. This means that there is a unique average $ \bar{A} $ at which point the continuation value equals the early exercise. Let $ j_0 $ be the largest index such that $ A_{i,j}^{j_0} < \bar{A} $. The new set of singular points becomes
		\begin{equation*}
			\left\lbrace  \left( A_{i,j}^{1}, P_{i,j}^{1} \right), \left( A_{i,j}^{j_0}, P_{i,j}^{j_0} \right), \left( \bar{A}, \bar{A} - K \right), \left( A_{i,j}^{\max}, A_{i,j}^{\max} - K \right)  \right\rbrace
		\end{equation*}
	\end{enumerate}
\end{enumerate}

We can repeat the same procedure iteratively for $ i = n-2, \dots, 0 $. At the final step, we get only one singular point, which gives us $ P_{0,0}^1 $, the exact American binomial price relative to the tree with $ n $ steps.

[TODO: Insert Figure 4 -- Not so important.]


\begin{rem}[Computational complexity -- European vs American]
	If we do not approximate, the number of singular points in the American case can never be greater than the European case. Thus, the American procedure cannot be slower than the European one. In case we also take approximations into account, the American case is expected to be faster heuristically, but cannot be proved in general.
\end{rem}

\begin{rem}[Put]
	For an Asian put, the exact same procedure has to be followed.
\end{rem}

\begin{rem}[Floating]
	In this case, we modify the procedure as follows: at maturity the singular points depend not on the strike $ K $ but rather on the underlying value at each node $ S_{i,j} $. Therefore the new singular points are obtained by replacing $ K $ by $ S_{i,j} $. The backward procedure is the same as before, just properly taking into account the new intrinsic values.
\end{rem}


\section{Approximation}
\label{sec:asian-approx}
The above sections introduced the singular points method, a procedure to evaluate the exact binomial price of an Asian option. Since for any node $ N_{i,j} $, we have that $ L_{i,j} \le L_{i+1,j} + L_{i+1,j+1} - 2 $, the resulting algorithm has exponential complexity, same as that of the binomial method.

Where this method shines, though, is its ability to use approximations to drastically reduce the order of complexity from exponential to polynomial time. The singular points method can be used to obtain upper and lower bounds of the binomial price at a fraction of the computational cost. Moreover, we can specify \emph{a priori} bounds on the error. We shall see that these are just simple consequences of Lemmas \ref{lmm:asian-upper-estimate} and \ref{lmm:asian-lower-estimate}.

\subsection{Upper bound}
\label{subsec:asian-ub}
In order to obtain an upper bound, we remove selected points from each node. That this is a upper estimate of the exact binomial price is guaranteed by Lemma \ref{lmm:asian-upper-estimate}.

Removal of points may be done on the basis of various criteria. One such method is as follows.

Consider the set of singular points $ C $ associated with the node $ N_{i,j} $ and the corresponding price value function $ v_{i,j} (A) $. Let $ v_{i,j}^u (A) $ be the price value function obtained by removing a point $ \left( A_{i,j}^{l}, P_{i,j}^{l} \right) $ from $ C $. We have
\begin{equation}
	\left| v_{i,j}^u (A) - v_{i,j} (A) \right|  \le  \epsilon_l  \qquad  \forall A \in \left[ A_{i,j}^{\min}, A_{i,j}^{\max} \right],
\end{equation}
where
\begin{equation}
	\epsilon_l  =  v_{i,j}^u \left( A_{i,j}^{l} \right) - v_{i,j} \left( A_{i,j}^{l} \right)  =  \frac{ P_{i,j}^{l+1} - P_{i,j}^{l-1} }{ A_{i,j}^{l+1} - A_{i,j}^{l-1} } \left( A_{i,j}^{l} - A_{i,j}^{l-1} \right) + \left( P_{i,j}^{l-1} - P_{i,j}^{l} \right) .
\end{equation}

Therefore, given any tolerance $ h > 0 $, we may remove a point $ \left( A_{i,j}^{l}, P_{i,j}^{l} \right) $ only if $ \epsilon_l < h $. Repeating this procedure sequentially at each node of the tree, while avoiding the elimination of two consecutive singular points, we can conclude that the upper estimate thus obtained differs from the exact binomial value by at most $ n h $.


\subsection{Lower bound}
\label{subsec:asian-lb}
The computational procedure is quite similar to the procedure of approximation using upper bounds. The theoretical foundation of this part is given by Lemma \ref{lmm:asian-lower-estimate}. If we remove the points $ \left( A_{i,j}^{l-1}, P_{i,j}^{l-1} \right) $ and $ \left( A_{i,j}^{l}, P_{i,j}^{l} \right) $, and add the point $ (\bar{x}, \bar{y} ) $ as described in Lemma \ref{lmm:asian-lower-estimate}, the new function is never greater than the original one, and the differences between the values at any point is bounded over by $ \delta_l $, where
\begin{equation}
	\delta_l  =  \frac{ P_{i,j}^{l} - P_{i,j}^{l-1} }{ A_{i,j}^{l} - A_{i,j}^{l-1} } \left( \bar{x} - A_{i,j}^{l-1} \right) + \left( P_{i,j}^{l-1} - \bar{y} \right) .
\end{equation}

Again, we allow for the operation only if $ \delta_l < h $.

An implementation methodology is as follows. At each node $ N_{ij} $, we consider the starting four points $ \left\lbrace  \left( A_{i,j}^1, P_{i,j}^1 \right), \left( A_{i,j}^2, P_{i,j}^2 \right), \left( A_{i,j}^3, P_{i,j}^3 \right), \left( A_{i,j}^4, P_{i,j}^4 \right)  \right\rbrace $. Calculate $ \delta_3 $. If $ \delta_3 < h $, we add the point $ (\bar{x}, \bar{y}) $ and remove the points corresponding to $ \left( A_{i,j}^2, P_{i,j}^2 \right) $ and $ \left( A_{i,j}^3, P_{i,j}^3 \right) $. The procedure will continue by considering the four new points $ \left\lbrace  (\bar{x}, \bar{y}), \left( A_{i,j}^4, P_{i,j}^4 \right), \left( A_{i,j}^5, P_{i,j}^5 \right), \left( A_{i,j}^6, P_{i,j}^6 \right)  \right\rbrace $ and proceeding in the exact same way. On the other hand, if $ \delta_3 \ge h $, then we do not remove any points and the procedure will continue by considering the four new points $ \left\lbrace  \left( A_{i,j}^2, P_{i,j}^2 \right), \left( A_{i,j}^3, P_{i,j}^3 \right), \left( A_{i,j}^4, P_{i,j}^4 \right), \left( A_{i,j}^5, P_{i,j}^5 \right)  \right\rbrace $. The procedure is repeated until all the singular points of the node has been taken into account.

\begin{rem}[Convergence to the exact binomial value]
	Jiang and Dai (2005) \cite{Jiang2004} proved the convergence of the exact binomial algorithm for European/American path-dependent options. In particular, they proved that the rate of convergence of the exact binomial algorithm to the continuous value is $ O( \Delta T ) $.
\end{rem}

The possibility of obtaining estimates of the exact binomial price with an error control allows us to prove easily the convergence of our method to the continuous value. By choosing $ h $ to depend on $ n $ so that $ n h(n) \to 0 $, we have that the corresponding sequences of upper and lower estimates converge to the continuous price value. Moreover, the choice $ h(n) = O(\frac{1}{n^2}) $ guarantees that the order of convergence is $ O( \Delta T ) $.

\begin{rem}[Compuational complexity -- theoretical considerations]
	The key issue in assessing the complexity of our algorithm lies in the upper and lower bound computation. Since the number of singular points eliminated depends on various factors, and we do not control the number of singular points directly, it is difficult to theoretically calculate the order of complexity of the algorithm. Nevertheless, the numerical results indicate
	that the present method is very competitive in practice.
\end{rem}



\clearpage
\section{The program}
\label{sec:asian-program}

\subsection{Algorithm}

\begin{algorithm}[H]
	\DontPrintSemicolon
	
	\KwIn{\\
		\qquad \emph{Contract details}  \\
		\qquad \quad time to maturity: $ T $, strike price: $ K $ \\
		\qquad \quad type: arithmetic/geometric, call/put, European/American, fixed/floating  \\
		
		\qquad \emph{Details of the underlying asset}  \\
		\qquad \quad initial price: $ s_0 $, volatility: $ \sigma $, continuous dividend rate: $ q $  \\
		
		\qquad \emph{Market parameters} -- spot interest rate: $ r $ \\
		
		\qquad \emph{Computational parameters} -- time steps: $ n $, error bound: $ h $ \\
	}
	
	\KwOut{The price of the option at the initial time}
	
	\Begin{
		Set $ \Delta T, u, p $ from the formulae in Section \label{sec:asian-binom}. \;
		
		Compute the singular points at maturity using Equation \ref{eq:asian-price-maturity-kin} and \ref{eq:asian-price-maturity-knotin}. \;
		
		\For{$ i \in \{ N-1, \dots, 0 \} $}{
			Evaluate $ P_{i,0}^1 $ and $ P_{i,i}^1 $ by Equation \ref{eq:asian-terminal-nodes} with the early exercise.
			
			\ForAll{$ N_{i,j}, \  j \in \{ 1, \dots, i-1 \} $}{
				Using Equation \ref{eq:asian-proj-up}, $\forall A_{i+1,j}^l, \forall l \in \{ 1, \dots, L_{i+1,j} \} $, compute $ B^l $. \;
				
				$ \forall B^l \in \left[  A_{i,j}^{\min}, A_{i,j}^{\max}  \right] $, compute $ v_{i,j}^c ( B^l ) $ by Equations \ref{eq:asian-up-lint} and \ref{eq:asian-up-pr}. \;
				
				Using Equation \ref{eq:asian-proj-dn}, $\forall A_{i+1,j+1}^l, \forall l \in \{ 1, \dots, L_{i+1,j+1} \} $, compute $ C^l $. \;
				
				$ \forall C^l \in \left[  A_{i,j}^{\min}, A_{i,j}^{\max}  \right] $, compute $ v_{i,j}^c ( C^l ) $ by Equations \ref{eq:asian-dn-lint} and \ref{eq:asian-dn-pr}. \;
				
				Sort the averages $ \{ B^l \}_l \cup \{ C^l \}_l \in \left[  A_{i,j}^{\min}, A_{i,j}^{\max}  \right] $ to obtain the set of $ L_{i,j} $ singular points. \;
				
				Compute the American price according to Section \ref{sec:fixed-strike-am} and obtain a new set of singular points with a new cardinality denoted, for simplicity, by $ L_{i,j} $ again. \;
				
				\Switch(Approximation){upper or lower}{
					\Case{upper bound} {Follow Section \ref{subsec:asian-ub} \;}
					\Case{lower bound} {Follow Section \ref{subsec:asian-lb} \;}
				}
			}
		}
		\KwRet{$ P_{0,0}^1 $    \tcp*{the upper or lower estimate of the exact binomial price with error smaller that $ nh $.} }
	}
	
	\caption{Pricing cliquet options using the singular points method}
\end{algorithm}

\paragraph{Analysis of algorithm}
[TODO if time permits] Consider computational complexity, space complexity.


\subsection{Implementation}
The algorithm was implemented in Python 3.5.0 (2015-09-13).

<TODO: The actual implementation goes here>
%\inputminted[tabsize=2]{python}{../code/tr_asian_singularpoints.py}
%\label{lst:asian}



\section{Extensibility}
\label{sec:asian-extensions}

Let us recapitulate the conditions required for the singular points method to work in the case of Asian options with arithmetic mean.
\begin{itemize}
	\item The ability to calculate the upper and lower bounds of the mean for all nodes of the tree.
	\item The recombinant nature of the tree for the underlying. Note that the tree for the option prices are \emph{not} recombinant.
	\item Convexity and piecewise-linearity of the price function on the mean of the underlying.
	\item Constant volatility
\end{itemize}

Keeping these in mind, let us look at the possibility of extending the singular points method to the following cases:
\begin{enumerate}
	\item Asian options with geometric mean and fixed volatility.
	\item Asian options with arithmetic mean and local volatility.
\end{enumerate}



\paragraph{Geometric mean and fixed volatility}
In the case of geometric options, we have a closed form formula under the Black-Scholes market model. Let us try to extend the singular points method to this case.

Firstly, we show that the result about the maximum and minimum paths still hold in the geometric case.

\begin{dfn}[Geometric mean]
	The geometric mean of the risky asset's prices $ (S_i)_{i \in [n]} $ is given by:
	\begin{equation}
		\label{eq:gm}
		G_{n} = \left( \prod_{i=0}^n S_i \right) ^{\frac{1}{n+1}}
	\end{equation}
\end{dfn}


\begin{lmm}
	At each node $N(i,j)$, the following hold:
	\begin{enumerate}
	\item The maximum average possible $ G_{i,j}^{\max} $ is attained by the path corresponding to the path with $j$ up movements followed by $(i-j)$ down movements.
	\item The minimum average possible $ G_{i,j}^{\min} $ is attained by the path corresponding to the path corresponding to the path with $(i-j)$ down movements followed by $j$ up movements.
	\end{enumerate}
\end{lmm}

\begin{proof}
	The proof is the same as Corollary \ref{crr:asian-up-dn-path}, with $A$ replaced by $G$ and relevant modifications.
\end{proof}


One of the central ideas behind the singular points method is that the price of the option is a convex, piecewise-linear function of the average $A$. But in the geometric case, this no longer holds true. For example, take a node $N_{i,j}$ with $ i = n-1 $. The price function given by $ v_{i,j}(G) $, with $ G \in [G^{min},G^{max}] $, can be calculated by the discounted expectation value.
\begin{align}
	v_{i,j}(G) &= \frac{1}{R} \left[ p v_{i+1,j+1}(G_u) + (1-p) v_{i+1,j}(G_d) \right] \\
	G_u &= \left( G^{i+1} S_0 u^{-i+2j+1} \right)^{\frac{1}{i+2}} \propto G^{\frac{i+1}{i+2}} \\
	G_d &= \left( G^{i+1} S_0 u^{-i+2j-1} \right)^{\frac{1}{i+2}} \propto G^{\frac{i+1}{i+2}}
\end{align}
Clearly, the final function $ v_{i,j} $ is not linear in $G$. Rather it is piecewise-concave. Thus we cannot use the singular points method in this case.


\paragraph{Arithmetic mean with local volatility}
In this case, the tree for the underlying is not recombinant, so we do not have more than one singular point in one (non-recombining) node. Essentially, we cannot use the singular points method for local volatility models.



\section{Results and conclusion}
\label{sec:asian-results}

[TODO: The results section will be very similar in format to the one in the Cliquet chapter. Sorry for not being able to specify this exactly.]


Thus, we have seen that the singular points method fails to be generalised, its shortcomings being the inability to deal with Asian options with geometric mean and to deal with local volatility cases. Nevertheless, it is quite efficient in its domain. It is comparatively fast, having experimental order of complexity $ O(n^3) $. It also allows specification of \emph{a priori} error bounds, allowing us to immediately prove convergence of order $ O(\Delta T) $ to the continuous value. Numerical results reported by \cite{Gaudenzi2010} shows that the method outperforms alternative algorithms, and may be viewed as an improvement on previous tree methods.


%%% Local Variables:
%%% mode: latex
%%% TeX-master: t
%%% End:


\chapter[Cliquet options]{The Singular Points method applied to Cliquet options}
\label{cha:sp-cliquet}
\input{tex/cliquet.tex}

\chapter{Epilogue}
\label{cha:epilogue}
%% !TeX root = ../thesis.tex
% !TeX spellcheck = en_GB
% !TeX encoding = UTF-8


In the thesis, our main focus was to develop the ideas behind the pricing of options, and to study two particular classes of path-dependent exotic options, namely Asian options and cliquet options. We saw that these options do not lend themselves to be priced under the Black-Scholes framework, and we have to resort to alternative ways of pricing them. They can be priced in the Cox-Ross-Rubinstein model, but the complexity is exponential with respect to the number of time steps, and is infeasible in practice. The singular points method is a new lattice based method which has the same theoretical complexity as the Cox-Ross-Rubinstein model. But it excels in allowing for approximations, and this reduces the complexity from exponential time to polynomial time of very low orders.

Albeit similar in the basic structure, the theory of how the singular points method may be applied for Asian options and cliquet options are quite different, as are the corresponding algorithms. On the one hand, in the case of Asian options, we saw that the algorithm neither did generalise for Asian options with geometric mean, nor for local volatility models. Moreover, the pre-existing algorithms (some lattice-based) were quite competitive to this method.

On the other hand, for cliquet options, the method was able to handle cases of variable rates of interest, volatility, and local caps and floors with ease. The algorithm is extremely fast in this case, and does in fact outdo most other competing algorithms in terms of simplicity, ease of implementation and low memory requirement. Furthermore, this seems to be one of the very few tree lattice-based methods available to price cliquet options.

As a further research, it might be interesting to explore on the possibility of customising the method for other exotic option types. It would also be interesting to conduct a further study on the theoretical complexity of the algorithm, especially the dependence of the number, closeness and removability of singular point with respect to initial data ($ S_0, K, T, \sigma, r, F_{loc}, C_{loc}, F_{glob}, C_{glob} $) and the computational parameters ($ m, h $).

In conclusion, the method is quite specific; its implementation depends very much on the type of option. Nevertheless, it is a significant leap in the domain of tree methods because of its inherent advantages over other methods.


%%% Local Variables:
%%% mode: latex
%%% TeX-master: t
%%% End:


\appendix


\backmatter
% List of figures (can be in the backmatter too)
%\listofalgorithms

\printbibliography


\end{document}



%%% Local Variables:
%%% mode: latex
%%% TeX-master: t
%%% End:
